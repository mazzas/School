\documentclass{beamer}


\usetheme{Warsaw}
\usecolortheme{crane}


\title{Groups \& Vector Spaces}
\subtitle{Mathematical Methods in the Physical Sciences}
\author{Steve Mazza}
\institute[Naval Postgraduate School]
{
  Naval Postgraduate School \\
  Monterey, CA \\
  \includegraphics[height=3cm]{images/NPS_logo.jpg}
}
\date {SE3030, Winter/2014 \\ Quantitative Methods of Systems Engineering}
\subject{Quantitative Methods of Systems Engineering}


\begin{document}

\frame{\titlepage}


\begin{frame}{Introduction}
    %TODO: put content here.
\end{frame}

\begin{frame}{Definition}
    A group is a set of elements, $G$, together with a set operation, $\cdot$, that satisfies the following conditions:
    \begin{block}{Group Conditions}
        \begin{description}
            \item [Closure: ] $\forall ~a, b \in G, a\cdot b \in G$
            \item [Association: ] $\forall ~a, b, c \in G, (a\cdot b)\cdot c = a\cdot(b\cdot c)$
            \item [Identity: ] $\exists$ exactly 1 element, $i \in G \mid \forall ~a \in G, i\cdot a = a\cdot i = a$
            \item [Inversion: ] $\forall ~a \in G ~\exists ~b \mid a\cdot b = b\cdot a = i$, where $i$ is the identity element.
        \end{description}
    \end{block}
\end{frame}

\begin{frame}{This is a test-a-roonie}
    \framesubtitle{Another test to see how Vim handles \LaTeX.}
    \begin{block}{Block Test}
        \begin{align*}
            x &= sin(y) \\
              &= -cos(y)^2
        \end{align*}
    \end{block}
\end{frame}

\begin{frame}{Block Types}
   \begin{block}{This is a Block}
      This is important information
   \end{block}
 
   \begin{alertblock}{This is an Alert block}
   This is an important alert
   \end{alertblock}
 
   \begin{exampleblock}{This is an Example block}
   This is an example 
   \end{exampleblock}
\end{frame}

\end{document}
