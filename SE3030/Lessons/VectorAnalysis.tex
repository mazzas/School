\documentclass{beamer}


\usetheme{Warsaw}
\usecolortheme{crane}


\title{Vector Analysis}
\subtitle{Mathematical Methods in the Physical Sciences}
\author{Steve Mazza}
\institute[Naval Postgraduate School]
{ 
    Naval Postgraduate School \\
    Monterey, CA \\
    \includegraphics[height=3cm]{images/NPS_logo.jpg}
}
\date {SE3030, Winter/2014 \\ Quantitative Methods of Systems Engineering}
\subject{Quantitative Methods of Systems Engineering}


\begin{document}

\frame{\titlepage}


\frame{{Introduction}
    We will extend our discussion of vectors from chapters 3, 4, \& 5 with the following broad overview of topics
    \begin{itemize}
        \item Vector products
        \item Differentiation
        \item Integration
    \end{itemize}
    The end of the chapter contains discussions of vector theorems, which we will save for another lecture.
}


\frame{{Applications of Vector Multiplication}
    We can apply the dot and cross products introduced in Chapter 3, Section 4, to calculate
    \begin{description}
        \item [Work: ] $Fd\text{ cos}\theta = \vec{F}\cdot\vec{d}$
        \item [Torque: ] $rF\text{ sin}\theta = \vec{r}\times\vec{F}$
        \item [$\angle$ Velocity: ] $\omega r\text{ sin}\theta = \lvert\vec{\omega}\times\vec{r}\rvert$
    \end{description}

    Angular velocity is solved by considering that the linear velocity $\vec{v}$ of some point $P$ is equal to $\vec{\omega}\times\vec{r}$, and that the magnitude of $\vec{v} = \lvert\vec{\omega}\times\vec{r}\rvert$, which is also equal to $\omega r\text{ sin}\theta$.
}


\frame{{Triple Products}
    \begin{block}{Triple Scalar Product}
        \[\vec{A}\cdot\left(\vec{B}\times\vec{C}\right)\]
    \end{block}
    \begin{itemize}
        \item Can be thought of as the volume of a parallelepiped.
        \item Is the determinant of a $3\times3$ matrix.
        \item The product is invariant under a circular shift:
            \begin{exampleblock}{Circular Shift Invariant}
                \[
                    \vec{A}\cdot\left(\vec{B}\times\vec{C}\right) = 
                    \vec{B}\cdot\left(\vec{C}\times\vec{A}\right) = 
                    \vec{C}\cdot\left(\vec{A}\times\vec{B}\right)
                \]
            \end{exampleblock}
        \item Satisfies an equality under cross product negation:
            \begin{exampleblock}{Negative Cross Product}
                \[
                    \vec{A}\cdot\left(\vec{B}\times\vec{C}\right) = 
                    -\vec{A}\cdot\left(\vec{C}\times\vec{B}\right)
                \]
            \end{exampleblock}
    \end{itemize}
}


\frame{{Triple Products (continued)}
    \begin{block}{Triple Vector Product}
        \[\vec{A}\times\left(\vec{B}\times\vec{C}\right) = 
        \left(\vec{A}\cdot\vec{C}\right)\vec{B} -
        \left(\vec{A}\cdot\vec{B}\right)\vec{C}\]
    \end{block}
    \begin{itemize}
        \item Known as the "BAC-CAB" product.
        \item Useful for simplifying some calculations in physics.
        \item Is anticommutative: 
            \begin{exampleblock}{Anticommutative}
                \[\left(\vec{A}\times\vec{B}\right)\times\vec{C} = -\vec{C}\times\left(\vec{A}\times\vec{B}\right)\]
            \end{exampleblock}
        \item Satisfies Lagrange:
            \begin{exampleblock}{Jacobi Identity}
                \[
                    \vec{A}\times\left(\vec{B}\times\vec{C}\right) +
                    \vec{B}\times\left(\vec{C}\times\vec{A}\right) +
                    \vec{C}\times\left(\vec{A}\times\vec{B}\right) =0
                \]
            \end{exampleblock}
    \end{itemize}
}


\frame{{Triple Products (continued)}
    \begin{exampleblock}{6.3.2}
        \begin{align*}
            \text{Applying: } W &= Fd\text{ cos}\theta \\
                                &= F\cdot d \\
            \text{Find: }
            \vec{B}\cdot\vec{C} &= B_iC_i+b_jC_j+B_kC_k \\
                                &= 2\cdot0 + (-1)\cdot1 + 3\cdot(-5) \\
                                &= 0 + (-1) + (-15) \\
                                &= 0 - 1 - 15 \\
                                &= -16
        \end{align*}
    \end{exampleblock}
}


\frame{{Differentiation of Vectors}
    $\frac{d}{dt}\vec{A}$ is the vector whose components are the derivatives of the components of $\vec{A}$.
    \begin{block}{Definition}
        \[\dfrac{d\vec{A}}{dt}=
        \cfrac{dA_x}{dt}\hat{i} +
        \cfrac{dA_y}{dt}\hat{j} +
        \cfrac{dA_z}{dt}\hat{k}\]
    \end{block}
}

\frame{{Differentiation of Vectors (continued)}
    \begin{exampleblock}{6.4.3}
    	% Consider problem 6.4.3
    \end{exampleblock}
}


\frame{{Fields}
    %TODO
}


\frame{{Gradient}
    %TODO
    % Consider problem 6.6.5
}


\frame{{Expressions Involving $\nabla$}
    %TODO
    % Consider problem 6.7.5
}


\frame{{Line Integrals}
    %TODO
}


\begin{frame}{Questions?}
	\begin{center}
		\includegraphics[width=.7\textwidth]{images/fin.png}
	\end{center}
\end{frame}

\end{document}
