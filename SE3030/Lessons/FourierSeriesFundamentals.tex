
\documentclass{beamer}


\usetheme{Warsaw}
\usecolortheme{crane}


\title{Fourier Series Fundamentals}
\subtitle{Mathematical Methods in the Physical Sciences}
\author{Steve Mazza}
\institute[Naval Postgraduate School]
{ 
    Naval Postgraduate School \\
    Monterey, CA \\
    \includegraphics[height=3cm]{images/NPS_logo.jpg}
}
\date {SE3030, Winter/2014 \\ Quantitative Methods of Systems Engineering}
\subject{Quantitative Methods of Systems Engineering}


\begin{document}

\frame{\titlepage}


\frame{{Introduction}
    Fourier series are like power series but are only used to represent periodic functions.
    \begin{center}
        \includegraphics[scale=0.4]{images/fourier.jpg}
    \end{center}
}


\frame{{Periodic Functions}
    \begin{block}{Simple Harmonic Motion}
        An object executing simple harmonic motion if its displacement from equilibrium can be written as $A \text{ sin } \omega t$ or $A \text{ cos }\omega t$ or $A\text{ sin }\left( \omega t+\phi \right)$.
    \end{block}
    \begin{center}
        \includegraphics[height=4cm]{images/simple-harmonic-motion.jpg}
    \end{center}
}


\frame{{Periodic Functions (continued)}
    The $x$ and $y$ components are $\left( A\text{ cos }\omega t, A\text{ sin }\omega t \right)$.  In the complex plane this could be rewritten as
    \begin{align*}
        z &= x+iy \\
        &= A\left( \text{ cos }\omega t+i\text{ sin }\omega t \right) \\
        &= Ae^{i\omega t}
    \end{align*}

    The \emph{amplitude} is the maximum displacement from equilibrium and the \emph{period} is the time of one complete oscillation.
}


\frame{{Applications of Fourier Series}
    In application,
    \begin{itemize}
        \item Fourier series do not tend to converge as rapidly as power series.
        \item Fourier series can represent discontinuous functions.
    \end{itemize}
    Often applied to problems involving,
    \begin{itemize}
        \item Sound
        \item Light
        \item Radio waves
    \end{itemize}
}


\frame{{Applications of Fourier Series (continued)}
    %TODO: get some good example like Fourier's heat equation.
}


\frame{{Average Value of a Function}
    \begin{block}{Definition}
        \[
            \text{average of } f(x) \text{ on } \left( a,b \right) =
            \dfrac{\int_a^b f(x) dx}{b-a}
        \]
    \end{block}
    When the average of a function over a period of time is 0 then the average of the square of the function is often of interest.
    The average value over 1 period of $\text{sin}^2nx$ and $\text{cos}^2nx$ are the same:
    \begin{equation*}
        \dfrac{1}{2\pi}\int_{-\pi}^{\pi}\text{sin}^2nx~dx = \dfrac{1}{2\pi}\int_{-\pi}^{\pi}\text{cos}^2nx~dx = \dfrac{\pi}{2\pi} = \dfrac{1}{2}
    \end{equation*}
}


\frame{{Fourier Coefficients}
    %TODO
}


\frame{{Dirichlet Conditions}
    %TODO
}


\frame{{Complex Forms of Fourier Series}
    %TODO
}


\frame{{Other Intervals}
    %TODO
}


\begin{frame}{Questions?}
	\begin{center}
		\includegraphics[width=.7\textwidth]{images/fin.png}
	\end{center}
\end{frame}

\end{document}
