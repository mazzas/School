\documentclass[letterpaper,10pt]{article}

%\setlength{\parindent}{0in}
%\usepackage{fullpage} 
\usepackage{amsmath}
\usepackage{amssymb}
\usepackage{enumerate}
\usepackage{graphicx}
\usepackage[table]{xcolor}
\usepackage{dcolumn}
\oddsidemargin 0.0in
\textwidth 6.5in
\newcolumntype{.}{D{.}{.}{-1}}
\newcommand*{\myalign}[2]{\multicolumn{1}{#1}{#2}}

%opening
\title{Homework I}
\author{Steve Mazza}
%\date{}

\begin{document}
\maketitle

\section*{Problem 1.1.16}
Some number of particles, $a$ leaves $x=0$ and heads toward $x=1$.  If $r$ is the fraction of particles reflected back toward $x=0$ at $x=1$ (iteration $n=1$) then $ar$ particles are returned and $a(1-r)$ particles escape.  At iteration $n=2$ (back at $x=0$) we have $ar\times r$ particles or $ar^2$ particles reflected back and, consequently, $ar \times (1-r)$ particles escaping.  We write the sequence as
\[
	ar^0(1-r), ar^1(1-r), ar^2(1-r), ar^3(1-r),\dots
\]
and we can generalize this as 
\begin{equation}
	ar^{n-1}(1-r)
\end{equation}
We further notice that odd values of $n$ occur at $x=1$ and even values of $n$ occur at $x=0$, so we can write the sequence for each as
\begin{align*}
	x=0&: ar^{1}(1-r), ar^{3}(1-r), ar^{5}(1-r), ar^{7}(1-r),\dots \\
	x=1&: ar^{0}(1-r), ar^{2}(1-r), ar^{4}(1-r), ar^{6}(1-r),\dots
\end{align*}
which we can generalize as
\begin{align}
	x=0&: ar^{2n-1}(1-r) \\
	x=1&: ar^{2n-2}(1-r)
\end{align}

In general, the sums for these series is determined by the equation
\begin{equation*}
	S=\dfrac{a}{1-r}
\end{equation*}
So to sum the series at $x=0$ we substitute values $a=ar^{2n-1}$ $r=(1-r)$ and get
\begin{align*}
	S&= \dfrac{ar^{2n-1}}{1-(1-r)} \\
	&= \dfrac{ar^{2n-1}}{1-1+r} \\
	&= \dfrac{ar^{2n-1}}{r} \\
	&= ar^{2n-2}
\end{align*}
Summing the series at $x=1$ we substitute values $a=ar^{2n-2}$ $r=(1-r)$ and get
\begin{align*}
	S&= \dfrac{ar^{2n-2}}{1-(1-r)} \\
	&= \dfrac{ar^{2n-2}}{1-1+r} \\
	&= \dfrac{ar^{2n-2}}{r} \\
	&= ar^{2n-3}
\end{align*}

Since the particles begin at $x=0$ and head toward $x=1$ first, the largest fraction of particles which can escape at $x=0$ ($n=2$) is $\frac{1}{2}$.

\section*{Problem 1.6.27}

\section*{Problem 1.10.2}

\section*{Problem 2.4.12}

\section*{Problem 2.5.5}

\section*{Problem 2.5.41}

\section*{Problem 2.5.59}

\end{document}