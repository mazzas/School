\documentclass[letterpaper,10pt]{article}

%\setlength{\parindent}{0in}
%\usepackage{fullpage} 
\usepackage{amsmath}
\usepackage{amssymb}
\usepackage{enumerate}
\usepackage{graphicx}
\usepackage[table]{xcolor}
\usepackage{dcolumn}
\oddsidemargin 0.0in
\textwidth 6.5in
\newcolumntype{.}{D{.}{.}{-1}}
\newcommand*{\myalign}[2]{\multicolumn{1}{#1}{#2}}

%opening
\title{Homework I}
\author{Steve Mazza}
%\date{}

\begin{document}
\maketitle

\section*{Problem 1.1.16}
Some number of particles, $a$ leaves $x=0$ and heads toward $x=1$.  If $r$ is the fraction of particles reflected back toward $x=0$ at $x=1$ (iteration $n=1$) then $ar$ particles are returned and $a(1-r)$ particles escape.  At iteration $n=2$ (back at $x=0$) we have $ar\times r$ particles or $ar^2$ particles reflected back and, consequently, $ar \times (1-r)$ particles escaping.  We write the sequence as
\[
	ar^0(1-r), ar^1(1-r), ar^2(1-r), ar^3(1-r),\dots
\]
and we can generalize this as 
\begin{equation}
	ar^{n-1}(1-r)
\end{equation}
We further notice that odd values of $n$ occur at $x=1$ and even values of $n$ occur at $x=0$, so we can write the sequence for each as
\begin{align}
	x=0&: ar^{1}(1-r), ar^{3}(1-r), ar^{5}(1-r), ar^{7}(1-r), \dots \dfrac{a(1-r)r}{(1-r^2)} \\
	x=1&: ar^{0}(1-r), ar^{2}(1-r), ar^{4}(1-r), ar^{6}(1-r), \dots \dfrac{a(1-r)}{(1-r^2)}
\end{align}
We recall
\begin{equation*}
	\sum\limits_{n=0}^\infty ar^n=\dfrac{a}{1-r} \mbox{\hspace{4em}}\forall x: |x|<1
\end{equation*}
So to sum the series at $x=0$,
\begin{align*}
	\sum\limits_{n=0}^\infty a(1-r)r^{2n+1} &= \sum\limits_{n=0}^\infty[a(1-r)r]r^{2^n} \\
	&= \dfrac{a(1-r)r}{1-r^2} \\
	&= \dfrac{a(1-r)r}{(1-r)(1+r)} \\
	&= \dfrac{ar}{1+r}
\end{align*}
Summing the series at $x=1$,
\begin{align*}
	\sum\limits_{n=0}^\infty a(1-r)r^{2n} &= \sum\limits_{n=0}^\infty[a(1-r)]r^{2^n} \\
	&= \dfrac{a(1-r)}{1-r^2} \\
	&= \dfrac{a(1-r)}{(1-r)(1+r)} \\
	&= \dfrac{a}{1+r}
\end{align*}

Since the particles begin at $x=0$ and head toward $x=1$ first, the largest fraction of particles which can escape at $x=0$ ($n=2$) is $\frac{1}{2}$.

\section*{Problem 1.6.27}
We apply the ratio test to $\sum\limits^{\infty}_{n=0}\frac{100^{n}}{n^{200}}$ as follows and determine that our series diverges since $\rho > 1$.
\begin{align*}
	\rho_{n} &= \left\lvert{\dfrac{\dfrac{100^{n+1}}{(n+1)^{200}}}{\dfrac{100^{n}}{n^{200}}}}\right\rvert \\
	\rho &= \lim_{n\to\infty}\left\lvert{\dfrac{\dfrac{100^{n+1}}{(n+1)^{200}}}{\dfrac{100^{n}}{n^{200}}}}\right\rvert \\
	&= \lim_{n\to\infty}\left\lvert \dfrac{100 n^{200}}{(1+n)^{200}}\right\rvert \\
	&= 100
\end{align*}

\section*{Problem 1.10.2}

\section*{Problem 2.4.12}

\section*{Problem 2.5.5}

\section*{Problem 2.5.41}

\section*{Problem 2.5.59}

\end{document}