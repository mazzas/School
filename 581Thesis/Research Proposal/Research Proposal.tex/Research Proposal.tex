%
%	Configure LaTeX to produce a PDF "book" using the memoir class
%

\documentclass[10pt,oneside]{memoir}

%
%	Generic Configuration for memoir-based documents
%

\usepackage{layouts}[2001/04/29]


% In case we need a glossary, or index
\usepackage{glossaries}
\glstoctrue
\makeglossaries
\makeindex


% Basic page layout configuration
\def\mychapterstyle{default}
\def\mypagestyle{headings}


% Use 8.5 x 11 inch page layout
%
%	8.5 x 11 layout for memoir-based documents
%


%%% need more space for ToC page numbers
\setpnumwidth{2.55em}
\setrmarg{3.55em}

%%% need more space for ToC section numbers
\cftsetindents{part}{0em}{3em}
\cftsetindents{chapter}{0em}{3em}
\cftsetindents{section}{3em}{3em}
\cftsetindents{subsection}{4.5em}{3.9em}
\cftsetindents{subsubsection}{8.4em}{4.8em}
\cftsetindents{paragraph}{10.7em}{5.7em}
\cftsetindents{subparagraph}{12.7em}{6.7em}

%%% need more space for LoF numbers
\cftsetindents{figure}{0em}{3.0em}

%%% and do the same for the LoT
\cftsetindents{table}{0em}{3.0em}

%%% set up the page layout
\settrimmedsize{\stockheight}{\stockwidth}{*}	% Use entire page
\settrims{0pt}{0pt}

\setlrmarginsandblock{1.5in}{1.5in}{*}
\setulmarginsandblock{1.5in}{1.5in}{*}

\setmarginnotes{17pt}{51pt}{\onelineskip}
\setheadfoot{\onelineskip}{2\onelineskip}
\setheaderspaces{*}{2\onelineskip}{*}
\checkandfixthelayout


% Use default packages for memoir setup
%
%	Default packages for memoir documents created by MultiMarkdown
%

\usepackage{fancyvrb}			% Allow \verbatim et al. in footnotes
\usepackage{graphicx}			% To enable including graphics in pdf's
\usepackage{booktabs}			% Better tables
\usepackage{tabulary}			% Support longer table cells
\usepackage[utf8]{inputenc}		% For UTF-8 support
\usepackage[T1]{fontenc}		% Use T1 font encoding for accented characters
\usepackage{xcolor}				% Allow for color (annotations)
\usepackage[sort&compress]{natbib} % Better bibliography support

\VerbatimFootnotes



% Configure default metadata to avoid errors
%
%	Configure default metadata in case it's missing to avoid errors
%

\def\myauthor{Author}
\def\defaultemail{}
\def\defaultposition{}
\def\defaultdepartment{}
\def\defaultaddress{}
\def\defaultphone{}
\def\defaultfax{}
\def\defaultweb{}


\def\mytitle{Title}
\def\subtitle{}
\def\keywords{}


\def\bibliostyle{plain}
% \def\bibliocommand{}

\def\myrecipient{}

% Overwrite with your own if desired
%\input{ftp-metadata}






\def\mytitle{}
\def\myauthor{}
\def\latexmode{memoir}
%
%	Get ready for the actual document
%

\usepackage[
	plainpages=false,
	pdfpagelabels,
	pdftitle={\mytitle},
	pagebackref,
	pdfauthor={\myauthor},
	pdfkeywords={\keywords}
	]{hyperref}
\usepackage{memhfixc}


%
%	Configure information from metadata for use in title
%

\ifx\latexauthor\undefined
\else
	\def\myauthor{\latexauthor}
\fi

\ifx\subtitle\undefined
\else
	\addtodef{\mytitle}{}{ \\ \subtitle}
\fi

\ifx\affiliation\undefined
\else
	\addtodef{\myauthor}{}{ \\ \affiliation}
\fi

\ifx\address\undefined
\else
	\addtodef{\myauthor}{}{ \\ \address}
\fi

\ifx\phone\undefined
\else
	\addtodef{\myauthor}{}{ \\ \phone}
\fi

\ifx\email\undefined
\else
	\addtodef{\myauthor}{}{ \\ \email}
\fi

\ifx\event\undefined
\else
	\date[\mydate]{\today}
\fi
\title{\mytitle}
\author{\myauthor}

\ifx\mydate\undefined
\else
	\date{\mydate}
\fi


\ifx\theme\undefined
\else
	\usetheme{\theme}
\fi

\begin{document}


\chapterstyle{\mychapterstyle}
\pagestyle{\mypagestyle}

% Frontmatter
\frontmatter

% Title Page
\maketitle
\clearpage

%
% Copyright Page
%

\vspace*{\fill}
\setlength{\parindent}{0pt}

\ifx\mycopyright\undefined
\else
	\textcopyright{} \mycopyright
\fi

\begin{center}
	\framebox{ \parbox[t]{1.5in}{\centering Formatted for \LaTeX \\ 
	by MultiMarkdown}}
\end{center}

\setlength{\parindent}{1em}
\clearpage

\tableofcontents
%\listoffigures
%\listoftables


\mainmatter


Thesis Proposal

Your Name
Degree and Program
Department
University Name

Committee Chair(s)

\_\_\_\_\_\_\_\_\_\_\_\_\_\_\_\_\_\_\_\_\_\_\_\_\_\_\_\_\_\_\_\_\_\_\_\_\_\_
Professor Runciter, Ph. D

\_\_\_\_\_\_\_\_\_\_\_\_\_\_\_\_\_\_\_\_\_\_\_\_\_\_\_\_\_\_\_\_\_\_\_\_\_\_
Professor Jefferson, Ph. D

Committee Member(s)

\_\_\_\_\_\_\_\_\_\_\_\_\_\_\_\_\_\_\_\_\_\_\_\_\_\_\_\_\_\_\_\_\_\_\_\_\_\_
Professor Weishaupt, Ph. D

\_\_\_\_\_\_\_\_\_\_\_\_\_\_\_\_\_\_\_\_\_\_\_\_\_\_\_\_\_\_\_\_\_\_\_\_\_\_
Professor Wilson, Ph. D

May 1, 2013

\pagebreak 

Abstract
Insert the abstract here. The abstract should summarize your research in clear and concise language. You should write your abstract with the idea that it may be the only section that will actually be read and the sole representation of your research. 

Some primary factors that you should consider for the abstract:

Motivation: Why do we care?
Problem statement: What are you trying to address?
Approach: How will you address the problem?
Results: What is the answer or expected answer?
Conclusions: What are the implications?

\pagebreak 

Introduction
Introduction to the research including background, current research in the area, etc.

\pagebreak 

Purpose or Hypothesis
The purpose or hypothesis of the research. This is a more general overview of your research.

\pagebreak 

Research Question(s)
Insert your specific and answerable research question(s) here. Bullet points are OK. The idea is to be as clear as possible.

\pagebreak 

Importance of the Project
The importance of the research including what makes your research unique and\slash or the specific contribution you seek to make.

\pagebreak 

Method of Approach
The research methodology you will employ including location, data collection techniques, analysis techniques, etc.

\pagebreak 

Work Plan
This is arguably the most important part of the proposal as it shows how you will conduct your study. As such, it should be as realistic as possible and very detailed. For my proposal I broke it down by what I intended to accomplish each month.

August
1. Determine survey questions

October
2. IRB request - 2 months
3. Meet with committee

November
4. Mail surveys
5. Data Entry
6. Submit abstract for professional conference paper or poster presentation

December
7. Data Entry

January
8. Data analysis and write-ups
9. Begin writing thesis draft
10. Brown bag presentation to department

February - April
11. Continue writing thesis and revise when necessary
12. Submit draft to committee for review 

May
13. Final draft complete by end of May with committee revisions

\pagebreak 

Budget
This should be as detailed as possible but simple to understand. A simple table with line items should suffice. Any descriptions or justifications can be done below the table.

Line Item
Amount

Total

\pagebreak 

References
Insert your references here. This includes pretty much everything read that has informed your research design. Make sure you use the format recommended by your discipline or organization. These examples are in Chicago.

RD Bullard, Dumping in Dixie: Race, Class, and Environmental Quality (Westview Pr, 2000).
RD Bullard, ``Environmental Justice in the 21st Century: Race Still Matters,'' Phylon (1960-) (2001): 151--71.
RL Bunch, and RE Lloyd, ``The Cognitive Load of Geographic Information,'' The Professional Geographer 58, no. 2 (2006): 209--20.
M Checker, Polluted Promises: Environmental Racism and the Search for Justice in a Southern Town (NYU Press, 2005).
M Granovetter, ``The Strength of Weak Ties: A Network Theory Revisited,'' Sociological theory 1 (1983): 201--33.
MS Granovetter, ``The Strength of Weak Ties,'' The American Journal of Semiotics 78, no. 6 (1973): 1360.
JK Jung, and S Elwood, ``Extending the Qualitative Capabilities of GIS: Computer-Aided Qualitative Gis,'' Transactions in GIS 14, no. 1 (2010): 63--87.
MP Kwan, and G Ding, ``Geo-Narrative: Extending Geographic Information Systems for Narrative Analysis in Qualitative and Mixed-Method Research,'' The Professional Geographer 60, no. 4 (2008): 443--65.

%
%	MultiMarkdown default footer file
%


% Back Matter
\if@mainmatter
	we're in main
	\backmatter
\fi


% Bibliography

\ifx\bibliocommand\undefined
\else
	\bibliographystyle{\bibliostyle}
	\bibliocommand
\fi



% Glossary
\printglossaries


% Index
\printindex



\end{document}
