\NPSappendixTOC{The NPS LyX Template}
The NPS LyX Dissertation Template was developed by CDR Michael
Bilzor is available for use with LyX. The template can be obtained from
\url{http://simson.net/npsthesis/lyxthesis.zip}.   The remainder of
this section was written by CDR Bilzor and describes using the
template.

Using the NPS LyX Dissertation Template

\section*{Get \LaTeX}
Download and install, if necessary, a \LaTeX{} installation package. For
Mac, I recommend using Mac\TeX{} - it includes some extras like BibDesk
and gives you a good basic \LaTeX{} 
editor. (\url{http://www.tug.org/mactex/2009/})

\section*{Get LyX}
Download it, install it, and tell it about your \LaTeX{}
  installation. (If the lack of a spell-checker is bugging you, an
  easy way to add one is via MacPorts (http://www.macports.org/) -
  once a package like aspell has been ported in, you can point LyX to
  it and it'll use it).

\section*{Editing your SF298}

Unfortunately the SF298 files provided by the \LaTeX{} system do not
work with LyX without modification. You therefore need to modify the
SF298 style file or use the one we have modified for you.

\subsection*{Option 1: modify the standard sf298 file (permanent fix)}

There are no Windows-specific directions for this task, but it has
been tested on MacOS and Ubuntu.

Find the copy of your \LaTeX{} distro's |sf298.sty| file:

\begin{description}
\item[MacOS]  	|/usr/local/texlive/2010/texmf-dist/tex/latex/sf298/sf298.sty|
\item[Ubuntu]   |/usr/share/texmf-texlive/tex/latex/sf298/sf298.sty|
\end{description}

These files are read-only, and may be hidden. To make the file
read-write, do a chmod at the command line (Unix gurus), or use Get
Info (Command-I) from the Finder and click the lock icon; you'll be
prompted for the admin password and can then make canges to the
permissions.

(The template includes  a modified copy of the |sf298.sty| file in the
|/styles| folder, in case you just want to set permissions, delete the
old one, and replace it with this one).

Open |sf298.sty| in a text editor (I recommend TextWrangler for Mac; I've
had issues with standard text editor in saving the file, but
TextWrangler was obedient).

Locate and comment out the following line (as shown below, now with a \% at the beginning):

\begin{Verbatim}
%\ExecuteOptions{config,nofloatlongboxes}
\end{Verbatim}

Skip the next line, then comment out the following three lines as shown:

\begin{Verbatim}
%\if Y\sf@config
%  \InputIfFileExists{sf298.cfg}{}{}
%\fi
\end{Verbatim}

Save the file and exit. Lyx will default to using the \LaTeX{}
distribution copy of |sf298.sty| unless you explicitly point it to the
location of a different copy of |sf298.sty| in your |\usepackage|
statement. Unmodified, you will get at least two errors compiling the
NPS dissertation file.  With |sf298.sty| modified, it will now compile
the NPS \LaTeX{} dissertation format without the errors.

\subsection*{Option 2: use the custom-named nps\_sf298 style file and keep a copy of it in your dissertation's working directory (easy fix).}

Open up the \LaTeX{} Preamble in LyX, using Document/Settings.  Once
there, change the statement |\usepackage{sf298}| to
|\usepackage{nps_sf298}|, and be sure there is always a copy of
|nps_sf298.sty| in your dissertation's working directory.

Open up the Template.  The file is called |Sample_nps_phd.lyx|.  It uses
the files in the |/images| directory, as well as the bibliography file
|Sample.bib|.  Typeset a PDF file to check it out.  If everything works,
you can use this as a template and edit your dissertation using the
LyX editor.

Notes: 
\begin{itemize}
\item This template takes most of the formatting from the |npsphd.cls|
  file (from Prof.\ Simson Garfinkel and MAJ Rob Harder), and adds
  the necessary parts into a LyX field called ``\LaTeX{} Preamble,'' which
  is accessible under document settings. The basic class is "report",
  and the master fonts are set to Times New Roman. To add more style
  files, add |\usepackage| statements to the ones already in the
  Preamble.

\item Be sure that the fields you use for Title, Name, etc. exactly match where they're entered in the document in multiple places (i.e., in the sf298 fields).

\item Sections like the Abstract, Acknowledgments, etc. are implemented as unnumbered chapters (chapter* in the pulldown).

\item |hyperref| support is available through the native LyX GUI interface.

\item The document class in LyX is just ``report,'' and all the customization comes from the \LaTeX{} preamble. LyX has a way to create what it calls a ``layout'' which is LyX-speak for a locally-defined custom \LaTeX{} document class. I haven't done this, but it's probably a good ``future work'' project.

\end{itemize}
