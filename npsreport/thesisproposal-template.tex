%%%%%%%%%%%%%%%%%%%%%%%%%%%%%%%%%%%%%%%%%%%%%%%%%%%%%%%%%%%%%%%%%%%%%%%%%
% Thesis Proposal Template
%
% This is a Latex template for the GSOIS thesis proposal.  It is modeled
% off of the Word template provided at http://www.nps.edu/Academics/Schools/GSOIS/Departments/CS/docs/THESIS%20PROPOSAL%20FOR%20CS%20and%20MOVES.doc.
%
% This should be built using the Makefile in the domex-report directory
% and adding thesisproposal-template.pdf (or the saved title) to the 
% ALL variable.
%
% Track Changes
% 2012-01-25 : Created (kmfoster)
%%%%%%%%%%%%%%%%%%%%%%%%%%%%%%%%%%%%%%%%%%%%%%%%%%%%%%%%%%%%%%%%%%%%%%%%%
\documentclass{article}
\usepackage{url}
\usepackage{ulem}
\usepackage{paralist}
\usepackage[letterpaper,hmargin=1in,vmargin=1in]{geometry}
\usepackage{bibentry}

\nobibliography*

\begin{document}
\noindent MEMORANDUM \hfill Date: [DATE]\\

\noindent 
From: [AUTHOR]\\
Section(s): 368\\

\noindent
To: Program Officer, CDR Duane Davis, USN\\

\noindent
Via: 
\begin{compactenum}
\item Thesis Advisor:  [ADVISOR 1]
%\item Thesis Advisor:  [ADVISOR 2]
\item Academic Associate: Thomas W. Otani
\item Chairman, Computer Science: Peter J. Denning
\end{compactenum}

\noindent 
\\
Subj: THESIS PROPOSAL\\
\\
Encl: (1) Thesis Proposal\\
%     (2) Institutional Review Board (IRB) Form
\\

\begin{compactenum}
\item Tentative Title of Proposed Thesis: [TITLE]
\item General Area of Proposed Thesis Research:  FORMTEXT      
\item Enclosure (1) is the Thesis Proposal with a milestone plan
  (dates/events) for research and thesis completion.
\item I expect that my thesis will be UNCLASSIFIED .  If classified, I
  have read Chapter V of NAVPGSCOLINST 5510.2, and the NPS Research
  Admin web page (http://www.nps.edu/research/research1.html)
  concerning Classified Theses.
\item I reviewed the Institutional Review Board (IRB) web page
  concerning the use of humans in research
  (http://www.nps.edu/research/IRB.htm).  I am aware that if I use
  humans as subjects I must forward an IRB application via my thesis
  advisor before any data collection can begin as outlined in NPGSINST
  3900.4.
\item I anticipate the following travel or other extraordinary
  requirements: N/A.
%If you are expecting to travel for conferences or meetings, please enter information below and uncomment relevant fields.
%\begin{itemize}
%\item [EVENT 1] in [PLACE 1], [DATES 1].
%\item [EVENT 2] in [PLACE 2], [DATES 2].
%\item Blue Team Summit in Linthicum, MD, 29 February - 3 March, 2012.
%\item Privacy Law Scholars Conference in Washington D.C., 7- 8 June, 2012.
%\end{itemize}
\end{compactenum}

\noindent
\newcommand{\sigspace}{\uline{\hspace{2in}}}
\newcommand{\datespace}{\uline{\hspace{.75in}}}

\begin{tabular}{lll}
\\
% Single Student Thesis
& \sigspace &  \\
& Student Signature & \\
% For Joint Thesis, uncoment next two lines and comment out previous two lines
%& \sigspace & \sigspace  \\
%& Student Signature & Student Signature \\
\\
\\
1. Approved and Forwarded:  & \sigspace & \datespace \\
& Thesis Advisor & Date\\
\\
\\
2. Approved and Forwarded: & \sigspace & \datespace \\
& Academic Associate, CS & Date\\ 
\\
\\
3. Approved and Forwarded: & \sigspace & \datespace \\
& Chairman, CS Department & Date\\
\\
\\
4. Approved and Retained: & \sigspace & \datespace \\
& Program Officer, CS Department & Date\\
\end{tabular}
\newpage
\begin{center}
{\large\textbf{THESIS PROPOSAL}}
\end{center}

\def\thesection {\Alph{section}.}
% prevents indenting for paragraphs
\setlength{\parindent}{0pt} 
% puts 2 carraige returns between paragraphs
\setlength{\parskip}{2ex}

%%%%%%%%%%%%%%%%%%%%%%%%%%%%%%%%%%%%%%%%%%%%%%%%%%%%%%%%%%%%%%%%%%%%%%%%%
% A. General Information
%%%%%%%%%%%%%%%%%%%%%%%%%%%%%%%%%%%%%%%%%%%%%%%%%%%%%%%%%%%%%%%%%%%%%%%%%
\section{General Information}

\begin{enumerate}
\item Name:  [AUTHOR 1]
\item Email:  \verb|[AUTHOR 1 SID]@nps.edu|
\item Curriculum:  Computer Science (368)
\item Thesis  Advisor: [ADVISOR 1]
%\item Thesis  Advisor: [ADVISOR 2]
\item Academic Associate:  Thomas W. Otani
\item Chair, CS Department: Peter J. Denning
\item Date of Graduation:  [GRAD MONTH], [GRAD YEAR] 
\end{enumerate}

%%%%%%%%%%%%%%%%%%%%%%%%%%%%%%%%%%%%%%%%%%%%%%%%%%%%%%%%%%%%%%%%%%%%%%%%%
% B. Area of Research
%
% Identify the proposed area of research in brief terms and state the 
% proposed title of the thesis.  The exact title can be changed as the 
% research and writing progress. A more detailed discussion is required 
% in D below.
%%%%%%%%%%%%%%%%%%%%%%%%%%%%%%%%%%%%%%%%%%%%%%%%%%%%%%%%%%%%%%%%%%%%%%%%%
\section{Area of Research}

The proposed topic of the thesis is [GENERAL AREA OF RESEARCH].  

The title of this thesis will be \textbf{[THESIS TITLE]}.

%%%%%%%%%%%%%%%%%%%%%%%%%%%%%%%%%%%%%%%%%%%%%%%%%%%%%%%%%%%%%%%%%%%%%%%%%
% C. Research Questions
%
% Identify the primary research question and subsidiary research
% questions. The primary research question should be broad enough
% that it covers the entire spectrum of the research activity.  
% Subsidiary research questions subdivide the primary research question 
% into manageable research segments. This should be a very explicit 
% statement of the questions the research will seek to answer.  While 
% the questions may be redefined later as the research progresses, the 
% initial objectives should be made very specific.
%%%%%%%%%%%%%%%%%%%%%%%%%%%%%%%%%%%%%%%%%%%%%%%%%%%%%%%%%%%%%%%%%%%%%%%%%
\section{Research Questions}

Research questions being pursued in this work include:

\begin{itemize}

\item This is the first research question.

\item This is the second research question and a citation \cite{mm2}.

\item This is the first research question.

\end{itemize}

%%%%%%%%%%%%%%%%%%%%%%%%%%%%%%%%%%%%%%%%%%%%%%%%%%%%%%%%%%%%%%%%%%%%%%%%%
% D. Discussion
%
% Discuss briefly the background leading up to this study, the major 
% issues and concepts involved, the key problems related to this area, 
% policy and operational considerations and implications, and the 
% theoretical framework within which the study will be conducted.
%%%%%%%%%%%%%%%%%%%%%%%%%%%%%%%%%%%%%%%%%%%%%%%%%%%%%%%%%%%%%%%%%%%%%%%%%
\section{Discussion}

[DISCUSSION]

%%%%%%%%%%%%%%%%%%%%%%%%%%%%%%%%%%%%%%%%%%%%%%%%%%%%%%%%%%%%%%%%%%%%%%%%%
% E. Scope of Thesis
%
% Describe the main thrust of the study, what areas will be specifically
% investigated and what areas will be excluded.  Put boundaries around 
% the study.  Identify what the study will be (e.g., a case study, 
% implementation guide, development of a model, assessment of a model, 
% policy or management guide).  Discuss any limitations of the study.
%%%%%%%%%%%%%%%%%%%%%%%%%%%%%%%%%%%%%%%%%%%%%%%%%%%%%%%%%%%%%%%%%%%%%%%%%
\section{Scope of the Thesis}

[SCOPE]

%%%%%%%%%%%%%%%%%%%%%%%%%%%%%%%%%%%%%%%%%%%%%%%%%%%%%%%%%%%%%%%%%%%%%%%%%
% F. Methodology
%
% This section should explain the conceptual foundation or framework 
% within which the questions will be examined and the specific research 
% techniques that will be used to answer the questions.  In other words, 
% the study should be conducted within the context of some recognized and
% valid model of problem solving and should use appropriate methods of 
% collecting and analyzing data relevant to the problem.  For example, if 
% the problem were to determine the optimal replacement cycle for lights
% on navigational buoys, the conceptual foundation (model) might be the 
% present value of life cycle cost.  The research method might then 
% include collection of actual data on the amounts of all relevant costs, 
% statistical analysis to determine the expected value of such costs, and 
% discounting to determine the lowest present value of all alternative 
% life cycle cost patterns.  Describe the general kinds of information to 
% be used, the sources from which the data will be collected, and the 
% methodology to be used in collecting the data.  Be specific in 
% identifying the method(s) of research (e.g., questionnaires, interviews, 
% etc.) why the method(s) selected are appropriate, and what types of 
% individuals/organizations will be approached.  Discuss how any 
% limitations identified in Part E above will affect your methodology or 
% data sources. Briefly list the kinds of questions you intend to include 
% in your questionnaire or you expect to use during interviews.
%%%%%%%%%%%%%%%%%%%%%%%%%%%%%%%%%%%%%%%%%%%%%%%%%%%%%%%%%%%%%%%%%%%%%%%%%
\section{Methodology}

[METHODOLOGY]

%%%%%%%%%%%%%%%%%%%%%%%%%%%%%%%%%%%%%%%%%%%%%%%%%%%%%%%%%%%%%%%%%%%%%%%%%
% G. Chapter Outline
%
% Identify the tentative chapter headings and provide brief discussion 
% of chapter content.
%%%%%%%%%%%%%%%%%%%%%%%%%%%%%%%%%%%%%%%%%%%%%%%%%%%%%%%%%%%%%%%%%%%%%%%%%
\section{Chapter Outline}

\begin{enumerate}
\item \textbf{Introduction} - An introduction to the research area and problem of interest.  
\item \textbf{Previous Work} - A survey of similar work in the field.
\item \textbf{Methodology} - A discussion of the Methodology.
\item \textbf{Test Design and Implementation} - A description of the tests.
\item \textbf{Results} - A discussion of results.
\item \textbf{Conclusion and Future Work} - A discussion of lessons-learned and future work.
\end{enumerate}

%%%%%%%%%%%%%%%%%%%%%%%%%%%%%%%%%%%%%%%%%%%%%%%%%%%%%%%%%%%%%%%%%%%%%%%%%
% H. Schedule
%
% This is a tentative list of target dates for completion of the 
% successive states of the project.  You will not be held strictly to 
% this schedule; it is a means of conveying to others when you expect to 
% complete major milestones of the study.  It is important to recognize 
% that the various stages of the work must be done in a logical sequence 
% and that these various stages require different amounts of time.  
% (Note that this section of the proposal fulfills the requirements that
% a schedule be attached to Thesis Report No. 1.)  Give the dates during 
% which the various stages will be accomplished.  Your stages may differ 
% from the sample below.
%
%  1. Literature review:    
%  2. Draft thesis, Cover through Chapter 2, and initial list of references 
% <to advisor(s) to check background knowledge and problem statement, and 
% to approve proposed plan to conduct research>   
%  3. Draft thesis checked by thesis processor for format:    
%  4. Construct research design:    
%  5. Conduct research and any associated travel:    
%  6. Analyze data:    
%  7. Draft thesis to advisor(s):    
%  8. Final thesis submission for signatures:     
%
% The list of activities/milestones above may be modified as required; 
% however, enough fidelity must exist to give the reviewers a good idea 
% of how you plan to complete your thesis.
%%%%%%%%%%%%%%%%%%%%%%%%%%%%%%%%%%%%%%%%%%%%%%%%%%%%%%%%%%%%%%%%%%%%%%%%%
\section{Schedule}

\begin{itemize}
\item Literature review: [TARGET DATE]
\item Draft Chapters 1 - 3: [TARGET DATE]
\item ...
\end{itemize}

%%%%%%%%%%%%%%%%%%%%%%%%%%%%%%%%%%%%%%%%%%%%%%%%%%%%%%%%%%%%%%%%%%%%%%%%%
% I. Benefits of Study
%
% State the contribution expected from your research efforts, what 
% individuals/organizations will use the results of your thesis and 
% what problems/issues you feel will be addressed/resolved.
%%%%%%%%%%%%%%%%%%%%%%%%%%%%%%%%%%%%%%%%%%%%%%%%%%%%%%%%%%%%%%%%%%%%%%%%%
\section {Benefits of the Study}

[DISCUSSION OF THESIS CONTRIBUTION]

%%%%%%%%%%%%%%%%%%%%%%%%%%%%%%%%%%%%%%%%%%%%%%%%%%%%%%%%%%%%%%%%%%%%%%%%%
% J. Anticipated Travel/Funding Requirements
%
% List the locations to which you must travel to conduct your research, 
% anticipated length of stay and approximate amount required for travel, 
% per diem and miscellaneous expenses.  Be realistic in identifying your 
% needs and do so as early as possible.  Your advisor and Curricular 
% Officer can assist in finding sources of funds.
%%%%%%%%%%%%%%%%%%%%%%%%%%%%%%%%%%%%%%%%%%%%%%%%%%%%%%%%%%%%%%%%%%%%%%%%%
\section{Anticipated Travel/Funding Requirements}
There is currently no anticipated travel.
% If you anticipate traveling for conferences or meetings for your thesis, fill out the appropriate information in the table below and uncomment the related lines.  This section should be consistent with what was previously reported on the signature page.
%\begin{center}
%\begin{tabular}{|l|l|l|l|}
%\hline
%\textbf{Event} & \textbf{Location} & \textbf{Dates} & \textbf{Approximate cost} \\
%\hline
%[EVENT 1] & [LOCATION 1] & [DATES 1] & [EST. COST 1] \\
%\hline
%[EVENT 2] & [LOCATION 2] & [DATES 2] & [EST. COST 2] \\
%\hline
%\end{tabular}
%\end{center}

%%%%%%%%%%%%%%%%%%%%%%%%%%%%%%%%%%%%%%%%%%%%%%%%%%%%%%%%%%%%%%%%%%%%%%%%%
% K. Preliminary Bibliography
%
% Provide a listing of representative materials consulted during 
% preliminary literature search.  This should include references to the 
% problem or issue to be studied, the organizations or other context in 
% which it arises, and the research method(s) to be used.  It should 
% include any prior studies of a similar nature.  A minimum of three 
% references is required.  A Thesis/Dissertation can be one of the three; 
% the other two must then be non-thesis/non-dissertation references.  The 
% final bibliography will probably be much more extensive.
%%%%%%%%%%%%%%%%%%%%%%%%%%%%%%%%%%%%%%%%%%%%%%%%%%%%%%%%%%%%%%%%%%%%%%%%%
\section{Preliminary Bibliography}
% This is a hack to reduce the space between the section header and the 
% reference list.  Since the reference list has a blank title, it results 
% in extra space between the reference contents and the section header.
% parskip value of 0 was tested but still left a large gap. 
% You should comment this out if it is causing problems
\setlength{\parskip}{-.15in} 

% print the bibliography with no title
\renewcommand{\refname}{} % redefines the reference title to blank

% list all references that should be included but weren't cited
\nocite{mm2} % don't technically need because we cited mm2, but just for demonstration
%\nocite{[CITATION 1]}
%\nocite{[CITATION 2]}
%\nocite{[CITATION 3]}

\bibliographystyle{plain}
% include the .bib file where your references come from, only list the file root (e.g. thesis for thesis.bib)
\bibliography{thesis}
%\bibliography{thesis,[BIB 2]}

\end{document}


   



