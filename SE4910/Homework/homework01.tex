\documentclass[letterpaper,10pt]{article}

%\setlength{\parindent}{0in}
%\usepackage{fullpage} 
\usepackage{amsmath}
\usepackage{amssymb}
\usepackage{enumerate}
\usepackage{graphicx}
\usepackage[table]{xcolor}
\usepackage{dcolumn}
\oddsidemargin 0.0in
\textwidth 6.5in
\newcolumntype{.}{D{.}{.}{-1}}
\newcommand*{\myalign}[2]{\multicolumn{1}{#1}{#2}}

%opening
\title{Homework 1}
\author{Steve Mazza}
%\date{January 20, 2012}

\begin{document}
\maketitle

\section*{Problem 1}

\section*{Problem 2}
The problem describes a step-wise function as follows:
\begin{align}
	f(t) &= T, t\geq T \\
	f(t) &= t, 0 > t > T \\
	f(t) &= 0, t\leq 0
\end{align}
In order to obtain the Laplace transform, the functions must be made continuous.  There are at least two ways to achieve this.  The first is to integrate.
\begin{equation}
	\int_0^T te^{-st} \, \mathrm{d} t + \int_{T}^{\infty} Te^{-st}\, \mathrm{d}t
\end{equation}
The other is to apply the unit step function (let's use this method).
\begin{align}
	f(t) &= t(u(t)-u(t-T))+Tu(t-T) \\
	f(t) &= tu(t)-tu(t-T)+Tu(t-T) \\
	F(s) &= \frac{1}{s^{2}} + \frac{\mathrm{d}}{\mathrm{d}s}\left[e^{-Ts}\frac{1}{s}\right] + T\left(e^{-Ts}\frac{1}{s}\right) \\
	F(s) &= \frac{1}{s^{2}} - \frac{1 - e^{-Ts}\left(Ts+1\right)}{s^{2}} + \frac{T\left(e^{-Ts}\right)}{s} \\
	F(s) &= \frac{1-e^{-Ts}\left(Ts+1\right)}{s^{2}} + \frac{T\left(e^{-Ts}\right)}{s}
\end{align}

\section*{Problem 3}
The partial-fraction expansion is obtained in MATLAB as follows:
\begin{verbatim}
>> num = [1 5 6 9 30];
>> den = [1 6 21 46 30];
>> [r,p,k] = residue(num,den)
\end{verbatim}
\color{lightgray} \begin{verbatim}
r =

  -1.0812 + 1.7051i
  -1.0812 - 1.7051i
  -0.1154 + 0.0000i
   1.2778 + 0.0000i


p =

  -1.0000 + 3.0000i
  -1.0000 - 3.0000i
  -3.0000 + 0.0000i
  -1.0000 + 0.0000i


k =

     1
\end{verbatim} \color{black}
The corresponding formatted equation for the solution is
\begin{equation}
	F(s) = 1 + \frac{-1.0812+1.7051i}{s+1-3i} + \frac{-1.0812-1.7051i}{s+1+3i} + \frac{-0.1154}{s+3} + \frac{1.2778}{s+1}
\end{equation}

The inverse Laplace transform is obtained in MATLAB as follows:
\begin{verbatim}
>> syms s
>> F = (s^4+5*s^3+6*s^2+9*s+30)/(s^4+6*s^3+21*s^2+46*s+30);
>> ilaplace(F)
\end{verbatim}
\color{lightgray} \begin{verbatim}
 
ans =
 
(23*exp(-t))/18 - (3*exp(-3*t))/26 + dirac(t) - (253*exp(-t)*(cos(3*t) 
	+ (399*sin(3*t))/253))/117
 
\end{verbatim} \color{black}
The corresponding formatted equation for the solution is
\begin{equation}
	\frac{23e^{-t}}{18} 
	- \frac{3e^{-3t}}{26} 
	+ \delta(t) 
	- \dfrac{253e^{-t}\left(\dfrac{\cos(3t) + 399\sin(3t)}{253}\right)}{117}
\end{equation}
\emph{Please also see the corresponding MATLAB file for additional work.}

\section*{Problem 4}
\emph{Please also see the corresponding MATLAB file for additional work.}

\section*{Problem 5}
\emph{Please also see the corresponding MATLAB file for additional work.}

\section*{Problem 6}
\emph{Please also see the corresponding MATLAB file for additional work.}

\end{document}