\documentclass[letterpaper,10pt]{article}

%\setlength{\parindent}{0in}
%\usepackage{fullpage} 
\usepackage{amsmath}
\usepackage{amssymb}
\usepackage{enumerate}
\usepackage{graphicx}
\usepackage[table]{xcolor}
\usepackage{dcolumn}
\oddsidemargin 0.0in
\textwidth 6.5in
\newcolumntype{.}{D{.}{.}{-1}}
\newcommand*{\myalign}[2]{\multicolumn{1}{#1}{#2}}
\usepackage{lastpage} % for the number of the last page in the document
\usepackage{fancyhdr}
\pagestyle{fancy}
\fancyhf{}
\lfoot{Lab Group 6: Assignment 3}
\rfoot{Page \thepage\ of \pageref{LastPage}}

%opening
\title{Assignment 3}
\author{Lab Group 6 \\ { \small (Steve Mazza, George Palafox, Donna Ray)}}
\date{April 25, 2012}

% In Kerzner, do problems 10, 19, and 31 of Ch 5, and problems 13 and�14 of Ch 7. Be sure to explain your answers for Ch 7 questions. Single word answers do not provide enough information to show your understanding of the questions.  (less is more)

\begin{document}
\maketitle

\section*{Chapter 5}
\subsection*{Problem 10}
On large projects, some people become experts at planning while others become experts at implementation.  Planners never seem to put on another hat and see the problems of the people doing the implementation whereas the people responsible for implementation never seem to understand the problems of the planners.  

How can this problem be resolved on a continuous basis?

An effective way to solve this problem would be to allow planners see how their plans get implemented in the field.  Perhaps if they spend a little time with those people implementing their plans, then they can see firsthand that things don’t always result according to plan and that tweaks and quick modifications may have to be made to get a plan in place.  Also, perhaps if the people performing the implementation spent some time with the planners, then they would see the hurdles they would have to jump to get an effective plan in place.  Perhaps if the planners spent one day per week with the implementers, and vice versa, then each group would have a greater appreciation for what each does and the importance of their respective roles.  This would also have an added advantage in that it would serve as a basis for training planners to become implementers and vice versa.  This would augment the skills of all employees adding to their work experience.

\subsection*{Problem 19}
The company has just hired a fifty-four-year-old senior engineer who holds two masters degrees in engineering disciplines.  The engineer is quite competent and has worked well as a loner for the past twenty years.  This same engineer has just been assigned to the R\&D phase of your project.  You, as project manager or project engineer, must make sure that this engineer works as a team member with other functional employees, not as a loner. 

How do you propose to accomplish this?  If the individual persists in wanting to be a loner, should you fire him?

The first step is to allow the employee to work the R\&D effort and closely monitor the employee's performance to see if the employee will work as a \emph{loner}.   Monitoring can be achieved by getting feedback from the team members.  If the employee is working as a loner, other team members will complain because they won't be getting the support they may need from the loner employee and you will notice at that point in time that the employee is working as a loner.  At that time, you have to council the employee in a positive direction.  For example, you don't want to tell the employee that they have to work in a team effort or else they will be fired.  You probably want to tell the employee that their contribution is critical and others need him/her expertise to accomplish their tasks.  But for this to happen, they must work as a team.  Letting the loner employee know the benefits of working as a team would be a good idea.  Letting him/her know the negative impacts if they don't work as a team should also be pointed out.  

I think that the employee should not be fired especially if they were hired with that much experience.  If the employee continues to work as a loner, then you should council him/her showing her the negative impacts that have been sustained as a result of working as a loner.  He/she should clearly see the importance of working as a team.  Perhaps one should ask if there is a personal reason why he/she feels that they would work best as a loner.  Perhaps there exists a past experience in a previous job that forced them to think working alone is the best course of action.  If this doesn't work out then you should involve the functional manager or his/her superiors.  As a last resort, if all agree that this isn't working out, then the \emph{firing} option may be exercised.

\subsection*{Problem 31}
\begin{quotation}
Manager A is a department manager with 30 years of experience in the company.  For the last several years, he has worn two hats and acted as both project manager and functional manager on a variety of projects.  He is an expert in the field.  The company has decided to incorporate formal project management and has established a project management department.

Manager B, a 30-year-old employee with 3 years experience with the company, has been assigned as project manager.  In order to staff his project, manager B has requested from manager A that manager C (a personal friend of manager B) be assigned to the project as the functional representative.

Manager C is 26-year-old and has been with the company for 2 years.

Manager A agrees to the request and informs manager C of his new assignment, closing with the remarks, ``This project is yours all the way.  I don't want to have anything to do with it.  I'll be too busy with paperwork as the result of our new organizational structure.  Just send me a memo once in a while telling me what's happening.''

During the project kickoff meeting it becomes obvious to both manager B and manager C that the only person with the necessary expertise was manager A.  Without the support of manager A, the time duration for project completion could be expected to double.

This situation is ideal for role playing.

Put yourself in the place of managers A, B, and C and discuss the reasons for your actions.
\end{quotation}

\begin{itemize}
\item \textbf{Manager A:} I am a department manager with experience as a project and functional manager.  I have been tasked with the preparation of paperwork as the result of my company's new organizational structure.  I really don't have the time and energy to expend in micromanaging project efforts that should be dealt with the project manager.  The responsibility of handling all technical aspects of the project is that of the PM.  I do realize that I have the most experience and have the required expertise to finish the current project assigned to Manager B quickly.   However, I will not intervene and will hold the PM to an agreed schedule.  At this point, I have already agreed to Manager B's request and have supplied him with Manager C.  I assume that he has selected him due to his level of competence and not his personal relationship.  I concurred to this selection because I have trust in my PM.
\item \textbf{Manager B:} I am requesting that my department manager (Manager A) supply additional help.  I realized that Manager C won't be sufficient to meet the agreed schedule and will take twice the time to accomplish all the tasks for the project.  Since Manager A has the sufficient expertise to complete the effort according to schedule, I am requesting his direct involvement or I will need additional manpower to help with the effort.
\item \textbf{Manager C:} I think this effort I have been selected to work on will be very challenging but a great experience.  I feel that I will need some help to meet the required timelines.
\end{itemize}

\begin{quotation}How can this problem be overcome?\end{quotation}

This problem can be overcome by having Manager A request a full project plan from Manager B and state clearly the reasons why he needs more man power or more experienced folks.  Why can't he meet the current schedule with Manager C?  Have Manager A ask Manager B why he can't meet the project deadlines with Manager C.  If he can't then why did he select Manager C?  If the answers backed up with some level of evidence seem sensible, then have Manager A supply some extra experienced folks or have Manager A partake in the project part time - perhaps one day a week.  If the explanation by Manager B doesn't seem sensible, then it may be necessary for Manager A to stand his ground and have Manager B proceed with the man power he has since it is sufficient.  If Manager B refuses, then this issue may have to be raised to the upper-level management.

\begin{quotation}How do you get manager A to support the project?\end{quotation}

In order for Manager B to convince Manager A that he will need folk(s) with his level of expertise or an equivalent experienced employee Manager B will have to formulate a plan and show Manager A that he does indeed need a more seasoned employee to support his project.  Manager B will also have to apologize for not realizing the just having Manager C on board was not sufficient.

\begin{quotation}Who should inform upper-level management of this situation?\end{quotation}

The department manager should inform upper-level management of this situation if it can't be resolved at the department manager's level.

\begin{quotation}When should upper-level management be informed?\end{quotation}

Again, upper-level management will need to be informed if the department manager can't resolve the issue.  If Manager B doesn't present a sensible plan outlining the need for more experience folks then this needs to be raised to upper-level management.

\begin{quotation}Would any of your answers change if manager B and manager C were not close friends?\end{quotation}  

Yes, perhaps if Manager B and Manager C were not close friends then Manager B would have hired someone more experience for his project.  It is not recommended that Manager A come out and say that Manager B hired Manager C because of their existing relationship.  Manager A needs hard evidence that this is so.  But if Manager B had hired a more experience person then perhaps Manager B would have an employee with the experience necessary to accomplish the project under the proposed schedule and would not have the need to request either Manager A or a more experienced person.

\section*{Chapter 7}
\subsection*{Problem 13}
Situations common to matrix that can develop into conflicts identified below with recommended cure.
\begin{enumerate}[a)]
\item Compatibility and incompatibility issues 

Confront the individuals together.  Try to develop some common ground, but must start with listening and civility.  This way, even if they don’t trust each other, they have to find a way to work around each other.  
\item Power struggles break balance of power

Determine if the power struggles are enabled by lack of clearly delineated project processes that need to be more clearly articulated and defined.  Where such voids exist, develop the specifications for the processes and the measures associated with the processes.  If individuals are collectivizing power beyond their functional expertise, functional management should be involved to work out compromises that ensure the functional area expertise assigned to the project is not by-passed or subsumed by other functional areas.
\item Anarchy

Smoothing can be a solution if the source of anarchy is due to uncertainty and the conflicts are due to imprecision and unformulated plans and not due to an “us versus them” attitude.  If an overarching goal can be developed that gets everyone moving in one direction, smoothing may resolve these kinds of conflicts.  Forcing may be necessary and appropriate if the conflicts require a force greater than the root of the anarchy.  If the project is at stake and the state of the anarchy is driving the team farther apart a quick solution using force to gain some mutual respect that starts the project back towards common principles.
\item Grouptitis

Get the team to collaborate and ensure that everyone participates.  Certain individuals may be dominating the team’s inputs and decisions.  Make sure that everyone has a chance to make his or her contributions and that everyone is recognized.  Develop sub-tasks that can be worked by individuals or small teams of individuals that necessitates a more distinctive and unique approach to the problem that requires working out the interfaces and boundaries to a greater depth than a more generalized or unified team solution.
\item Collapse during economic crunch

Diversify and generalize roles in the team, since the resources may diminish that will require each team member to do more with less.  Generalizing the roles will require more reliance on each other, and may foster a can do spirit, that opens up conceptualizing in finding solutions to problems as they arise.  Problems have to be shared and continually worked since they could impact other project areas immediately or a solution that resolves the problem for one area turns into a bigger problem in another project area.
\item Decision Strangulation processes

Too much input and not enough output.  Make sure that decision making is not reserved to too small a set of individuals and define roles more specifically to make it easier to empower staff for decisions within their jurisdiction.  Facilitate communications between departments to make sure that decision reviews are based on impacts to other functions.  
\item Matrix organization goes too lower in the organizational structure

Use a hierarchal process to set performance standards and roles within the project office.  Performance standards cut-offs should define staff assignments between project office and functional support.
\item Navel gazing (internalized versus customer oriented)

Emphasize goals that measure project performance based on customer satisfaction in current markets and that expands product to new markets.  Formulate time based strategies for achieving these goals and determine end of life cycle for existing team and project endeavors.
\end{enumerate}

\subsection*{Problem 14}

\begin{table}[htdp]
\begin{center}
\begin{tabular}{p{3cm}ccccc}
\hline
\textbf{Conflict Resolution Mode} & 
\myalign{c}{\textbf{Confronting}} & 
\myalign{c}{\textbf{Compromising}} & 
\myalign{c}{\textbf{Smoothing}} & 
\myalign{c}{\textbf{Forcing}} & 
\myalign{c}{\textbf{Avoiding}} \\
\hline
Personality Issue (individuals) & & & X & & \\
Personality Issue (departments) & & X & & & \\
Responsibilities & X & & & & \\
\hline
\end{tabular}
\end{center}
\end{table}

\end{document}