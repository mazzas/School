\documentclass[letterpaper,10pt]{article}

\usepackage{lastpage} % for the number of the last page in the document
\usepackage{fancyhdr}
\pagestyle{fancy}
\fancyhf{}
\lfoot{Lab Group 6: Assignment 2}
\rfoot{Page \thepage\ of \pageref{LastPage}}
%\setlength{\parindent}{0in}
%\usepackage{fullpage} 
\usepackage{amsmath}
\usepackage{amssymb}
\usepackage{enumerate}
\usepackage{graphicx}
\usepackage[table]{xcolor}
\usepackage{dcolumn}
\oddsidemargin 0.0in
\textwidth 6.5in
\newcolumntype{.}{D{.}{.}{-1}}
\newcommand*{\myalign}[2]{\multicolumn{1}{#1}{#2}}

%opening
\title{Assignment 2}
\author{Lab Group 6 \\ { \small (Steve Mazza, George Palafox, Donna Ray)}}
\date{April 18, 2012}

\begin{document}
\maketitle

% SI3400 Homework #2:   Answer problems 3-9, 3-12, 3-26 and 3-39 in Kerzner.   
% For problem #3-26, what will have to change to optimize Ralph's implementation of a project management structure?
%  For problem #3-39, in additon to the questions posed, several of the project managers are leaving the company or being transferred out of their current positions, requiring selection of new PM�s. It is your job to develop criteria for senior management to use in the selection of these new PM�s. What qualities will you look for? What experience level? Will the selection be permanent, temporary, or probationary? What potential positions inside the company will you look to to find your new PMs? What are the pro's and con's of selecting an existing employee for the PM position? What are the pro's and con's of hiring someone for the PM position from outside the company? When hiring outsiders, how do you recommend the changes be made? 
%  What other factors should be considered in your plan?
%  Be sure to answer all the questions.
%  Remember to keep your submission to six pages or less as I will print out and read no more than the first six.

\section*{Problem 3-9}
% Should a company be willing to accept a project that requires immediate organizational restructuring?  If so, what factors should it consider?
The motivation for a company to accept a project at all should be objectively driven by how that project will affect the company's balance sheet.  The ways in which a company's financial health can be affected by a project are varied.  Among them are the costs and benefits of employee morale and turnover, costs and benefits of adjustments to market focus and profit centers, and the costs and benefits to systemic change.  Focusing on systemic change, or organizational restructuring, there are several factors that should be carefully considered.  These include proper organizational planning, proper training, and employee involvement.

Transition to a project-driven organizational structure requires acceptance and support by employees all up and down the corporate ladder but must begin with senior management.  Kerzner points out that, ``The strongest driving force of success during transition is a demonstration of commitment to and involvement in project management by senior executives.''\footnote{Kerzner, \emph{Project Management} (New York, NY: John Wiley \& Sons, Inc., 2009) 131.}

Authority of Project Managers during transition phases might not always be explicit.  There may be times when employees are not clear on where their supervision should be coming from.  Consequently, communication may begin to break down and management should be prepared for conflict resolution.  All of this increases the necessity for acceptance at all levels of and commitment to the organizational changes.

In conclusion, I believe that a company should be willing to accept a project that requires immediate organizational restructuring but should do so only after studying the various ways in which the project can influence the overall financial health of the company.  The principal concern should be for the impact to the company's balance sheet and not the intrinsic risks and rewards of the restructuring, itself.  Successful transition hinges on involvement and acceptance by employees at all levels and begins with senior management.

\section*{Problem 3-12}
% A major American manufacturer of automobile parts has a division that has successfully existed for the past ten years with multiple products, a highly sophisticated R\&D section, and a pure traditional structure.  The growth rate for the past five years has been 12\%. Almost all middle and upper-level managers who have worked in this division have received promotions and transfers to either another division or corporate headquarters.  According to ``the book,'' this division has all the prerequisites signifying that they should have a project organizational form of some sort, and yet they are extremely successful without it.  Just from the amount of information presented, how can you account for their continued success?  What do you think would be the major obstacles in convincing the personnel that a new organizational form would be better?  Do you think that continued success can be achieved under the present structure?

Despite the fact that this parts division is due for organizational change, they have managed to be extremely successful in their present form.  To explain some of their past and current success we should examine some of the advantages of the traditional organization.  The classical organization facilitates better budgeting and cost control, better technical control, more flexibility in the use of human resources (manpower), and tends to facilitate communication.  Kerzner also sites an increased capacity for quick reaction which would have served the parts division well in a highly competitive market.\footnote{Ibid, 96.}  So long as there is stability in management (both line and project) and needs are being met, continued success can be assumed.  Given, however, what we know about the recent management changes and the projected growth we can safely recommend a project organizational structure.

As with any company or division assessing organizational change, the major obstacles in convincing personnel that a new organizational form would be better hinge on several key points.  First and foremost, all employees should be keenly aware that the changes have the full endorsement and support of upper management.

Also, in any time of reorganization it is natural for employees to be concerned about job security.  Assuring staff that they will continue to be a valued member of the organization is a large but not insurmountable obstacle. This concern is often exacerbated when, for example, employees in a matrix organization find themselves reporting to multiple managers.  Assisting the employee in this situation can be helped by keeping the employee assigned to a single long-term project and providing clear guidance on tasking, evaluation, and formal job assessment.

Additionally, line and project managers must be willing to work together and lines of communication should be open both horizontally as well as vertically.  Conflict resolution should be handled swiftly by project managers and should only occasionally involve upper management.  To this end, any amount of management education and training that the division can afford is easily justifiable.  There is a considerable cultural shift as project managers learn to negotiate for resources and attempt to affect successful project completion sometimes with only implied authority to do so, especially during transitional periods.

I doubt that continued long-term success is likely under the present structure.  given the projected rate of growth and recent middle and upper-management turnover, the advantages of the traditional organization will quickly begin to to be supplanted by its shortcomings.

\section*{Problem 3-26}
In order to optimize Ralph's implementation of his management structure, assuming that other departments don't change their organization structure, then Ralph will have to get his folks trained in other areas such as logistics, finances, quality assurance, etc. so that when these departments are needed for the project to progress then the expertise is there. This will add more time and cost to the project since Ralph will have to invest in getting his folks trained. Perhaps a way to make things work will have to be for Ralph to implement pure product or projectized organizational structure but the drawback here is that there will be redundancy occurring between departments where the product being developed in Ralph's department may be similar or the same as one being developed in other department. Again, more time and cost incurred here. 

Another way to optimize Ralph's management structure may be to have other departments partially adopted some of Ralph's management structure processes. It may not be necessary to have each department have exactly the same management structure as Ralph's but if some processes are adopted, it may be sufficient to communicate horizontally to obtained the expertise needed to successfully execute the project. This may involve compromising with other departments to reach an agreement. This method is preferred other the former since time and cost may be saved as a result. Also, this will avoid redundancy. 

Either way, not having a similar management structure across departments will have a slight negative impact for the Project Manager.

\section*{Problem 3-39}
% For problem #3-39, in addition to the questions posed, several of the project managers are leaving the company or being transferred out of their current positions, requiring selection of new PM’s. It is your job to develop criteria for senior management to use in the selection of these new PM’s. What qualities will you look for? What experience level? Will the selection be permanent, temporary, or probationary? What potential positions inside the company will you look to to find your new PMs? What are the pro's and con's of selecting an existing employee for the PM position? What are the pro's and con's of hiring someone for the PM position from outside the company? When hiring outsiders, how do you recommend the changes be made? 
% What other factors should be considered in your plan?

As the management has seen, the current organizational structure of individual projects is not optimally suited to the subcontractor's multiple product lines that are nearly identical.  It is not an effective organizational structure and reorganizing the projects within a matrix organization would eliminate some duplication of effort.  In transforming to a matrix organization, maintaining the top level secrecy of the individual programs is a high priority of the matrix organization.  The need to maintain confidentiality of the individual programs will require the project managers to maintain strict control of the exchange of program information by staff members working in the matrix.  Before the organization can shift to a matrix type organization, management will need to develop a plan, or multiple plans, that very specifically addresses the secure exchange of program specific information in the matrix organization, and re-visit the plan regularly.  The plan should have the buy-off of each program manager and his or her customers.  Having the program managers and the line managers facilitate coordination meetings that include cross program discussions would be a means of facilitating direct exchange of information and keeping the exchanges focused and in control.  There should be regularly scheduled meetings with the programs' customers to inform them of the outcomes of these discussions and to keep the customer advised of any cost advantages due to shared resources.  The technology exchange controls add additional overhead to the programs' structures, but can be regarded as a necessary evil in order to facilitate controlled data exchanges that is not wide open but which does not stifle the exchange of technology that is advantageous to the individual programs of the organization.

%TODO: Insert graphic here.

When hiring a product manager for the matrix organization, qualities such as broad intellect, technical and programmatic training, introspective with perseverance to deal with plethora and prioritization of issues, and very good communication skills (listening, oratory, written) would serve the prospective candidate well in facing the challenges of the product manager position.  Effective PMs should be good at facilitating open communication; only by eliciting honest discussion will the program's real warts be exposed; simply not giving voice to issues will not mean they have been resolved, so the PM has to use his or her skills to foster an atmosphere of independent critical examination of the processes and performance internal to the project and externally.  

The candidate should possess as much experience as possible; not necessarily a wide variety of jobs, but having used and applied a variety of techniques to an array of project tasks that would similarly be expected to be encountered in the project, and that helped consummate successful results and accomplishments.  These project managers will perform as subcontractors in a high security environment, so typically the project manager would be expected to handle at least 2 levels of customers, the prime contractor and the government customer.  The PM would be responsible for ensuring the contract scope was well articulated and that it ultimately met the final customer's requirements.  The PM has the task of managing the entire effort's budget resources, and must be capable of soliciting and enabling accountability from each of the organization's functional support areas and management staff.

The project manager should be given a probationary period to determine whether his or her skills and interests are compatible with the specific characteristics that will result in successful performance within this project's organizational constraints and resources, including staff relations and project earned value performance.  

The product manager position will require a person that has had reasonable professional exposure to the demands of delivering a product or service on time and on budget and reporting the status of these matters to senior managers and having worked cross-functionally.  Production and test division, Engineering, Contracting and Cost management are principal areas that would have tested these abilities that would typically be necessary for a product manager.  

Hiring from within the organization will ensure that the product manager has familiarity with corporate strategy and policy and day to day business methods of the organization.  The candidates’ experience and personal qualities will also be well known and familiar to other staff members and management which can ease alignment of the organization to the new PM's personal style and manner.  The inside hire may tend to maintain status quo because of comfort zones, not because the process works well or can't be improved.  The insider may accept the status quo, without looking for actual performance data that quantifies the effectiveness of the processes.  The inside hire's lack of outside experience may limit his or her approaches to problems to those that have proven successful, but which may not be the best solution.

Expanding the base of potential candidates for the PM position will obviously widen the talent pool.  Hiring from outside the organization allows inclusion of candidates that have experiences outside the organization who can supplement the organization's tried and true approaches with varied approaches that open new opportunities to excel.  The outside hire is more likely to take a more critical view of each situation which can be refreshing to the organization, but may also be disruptive and non-productive leading to staff resentment of the management and disillusionment with their role in project.  

The organization's approach to introduction of outside hires for the PM position should give the new PM an opportunity to engage the project's line management thoughts about their business methods and use these discussions to determine the focus of the PM's role within their business model.  The management of the program must include plans to address how the project will be executed by the organization's functional teams.  The prospective product managers should conduct a review of all program plans.  The first step in the review cycle will be to determine how well the existing plans will fit with the transformation to a matrix organization, (e.g., if there are no plans, or voids in the plans, then these need to be created).  Staff from each function that has responsibilities for executing the plan should be included in the reviews.  Any necessary changes and modifications to processes will need the buy-in of the functional staff members, as the functional staff members are the best source of information on tailoring the plans that will become their responsibility to execute.

Other factors to include in selection of the PMs should include initiative, motivation, work ethos, humanity, deliberation, ability to present the truth and the ability to adapt.

\end{document}