\documentclass[letterpaper,12pt]{article}

%\setlength{\parindent}{0in}
%\usepackage{fullpage} 
\usepackage{amsmath}
\usepackage{amssymb}
\usepackage{enumerate}
\usepackage{graphicx}
\usepackage[table]{xcolor}
\usepackage{dcolumn}
\oddsidemargin 0.0in
\textwidth 6.5in
\newcolumntype{.}{D{.}{.}{-1}}
\newcommand*{\myalign}[2]{\multicolumn{1}{#1}{#2}}

%opening
\title{Assignment 1}
\author{Steve Mazza}
\date{April 11, 2012}

\begin{document}
\maketitle

\section{Case Study}
% A. (40pts) Describe and analyze a situation of mismanagement of which you have very good knowledge of both the events and the rationale for the various point of views. You may use your best guess to fill in obvious gaps in your knowledge. Please indicate your confidence level for each item that is not either gained from first- or second-hand sources to determine the validity of each perspective (Be a lawyer about this issue). You may feel more comfortable in giving the participants in your situation ‘stage’ or generic names such as ‘manager’ and ‘worker 1’ and ‘worker 2’. What were the problems you determined? How did they compound? What was the resulting damage? The objective is to see if you can describe the ‘obvious’ problem as well as the subtle problems.

% Once you have described the situation, the problems, and your observations, please describe how you would have handled that situation had you been the person making the decisions, and dealing with the workers. How would you have interpreted the issues differently, what additional information would you have gathered, and what counsel would you have sought before coming to your conclusions?

% NOTE: consider talking about the early phases of TM development (resource constraints, impossible schedule, weekly progress meetings, micromanagement, derisive language).
A little over a year ago I became involved in a rather large software engineering effort.  While the Division that I am in is experienced in managing the software development and project engineering process, most of the work has traditionally gotten done through our contractors.  This means that the focus of the management and oversight is on the contractual vehicle and not on the more traditional PM tasks.  

So acting as our sponsor, our Chief was left a little unclear as to what his role should be and he often erred on the side of asserting too much involvement in the details of implementation.  Worse, since he appeared not to fully recognize that we were both setting and meeting appropriate goals and deadlines, he felt the need to hold weekly progress reviews.  These often took the form of demonstrations and insisted on seeing working implementation of large parts of the system.

Several weeks into the project we had hurried a poorly designed version of enough of the project such that it would hold up in our lab under carefully controlled conditions. And we had yet to even complete our requirements gathering.  We were literally building to suit the demands of management to see progress while foregoing most of the other activities that were necessary for actual software engineering.  The process (such that you can refer to it as a \emph{process}) was at its worst when we found ourselves taking direction on the color and layout of the user interface from the project sponsor.

Fortunately we were rescued by the Second Law of Thermodynamics which states (loosely) that re-organizations will occur continuously throughout an organization until all of the talent is sufficiently diluted.  Under new sponsorship my team has been able to more than make up all of the ground that was lost.

When I look back what I see is an effective and well-meaning organization that has considerable experience in managing the efforts of defense contractors but that did not clearly know how to delegate and manage internally staffed projects.  That seemed to be coupled with a mistrust (lack of faith?) in the capabilities of government engineers.  The primary result was a considerable loss of time.  But more importantly the development team lost faith in our Division-level management, feeling that there was no support that would result in a successful execution.  At worst, my lead engineer once complained that he felt upper management was actually betting against our success.

After our reorganization I was able to assess the damage and salvage some of the work that had been done.  All of the product that satisfied the weekly sprints and reviews was tossed out for lack of value.  I was able to negotiate a new schedule and budget and have been given the latitude necessary to achieve broad success in our ongoing execution of this project.

I believe that the most damaging key factors in the early stages of this project revolved around the inappropriate involvement of upper management in the Project and Line Management activities.  I am relatively certain that this inappropriate involvement resulted from a lack of experience in managing internally staffed projects.  I am very grateful to our present leadership for their ongoing support and guidance which they provide on request without injecting themselves unnecessarily into the process.

\section{Article Assessment}
% B. (10pts) Think about the recommendations in the assigned article in Resources. This was found on LinkedIn - a networking website.

% How does this fit in with your perceptions of what a PM does and is responsible for? Does this sound like an experienced PM, or one that's just getting started? What tells you this?
Our author makes two basic recommendations, both of which are conceptually sound but which also expose our author's inexperience with large scale project management tools and practices.  

The first recommendation is that the Project Manager should communicate effectively with upper management.  The specifics of this recommendation are to use online tools.  Fair enough, supposing everyone has reasonable access to them.  But the recommendation goes even further to specify Google Docs, an online office suite comprised of a word processor, a spreadsheet, and presentation creation software.  While Google Docs can facilitate collaboration in a team environment for the creation of documents and spreadsheets, it is not well suited to large scale project management.  Even with the application of clever document templates, one would only hope to support small scale as-hoc project management at best.

The second recommendation is to implement a task structure that results in a reasonable sized work items.  There is nothing at all wrong with this approach and it is used by planners and project managers all the time.  What is somewhat flawed, however, is the author's assertion that the overall project timeline can be accurately estimated simply as a sum of the individual tasks.  For trivial projects with one or two team members, this approach might result in a successful estimation of time and cost but it would quickly break down under increased complexity.  Also, the assertion, ``\dots if there is something wrong with a sub-project then project[sic] is going to fail,'' belies the author's experience and depth of understanding of real world, complex project management.

\end{document}