\documentclass[letterpaper,10pt]{article}

%\setlength{\parindent}{0in}
%\usepackage{fullpage}
\oddsidemargin 0.0in
\textwidth 6.5in

%opening
\title{Homework for Module 2}
\author{Steve Mazza}

\begin{document}
\maketitle
Based on the analysis of preliminary system requirements it was determined that there is a need for additional security at Forward Operating Bases (FOB's) from those threats identified as direct and indirect fire, forceful and deceptive unlawful base penetration, and suicide bombings.  The threat is present and increasing with no abatement in sight.  This shall be developed as system to provide portable, effective, reliable, lethal force protection at small forward operating bases.  We can leverage existing highly effective systems like the Army's Counter-Rocket Artillery System (C-RAM).
\par
Principally, the customer needs an automated way to secure FOB's against enemy ground attacks.  They need a system that picks up where C-RAM leaves off.  We shall focus on defending against physical ground attacks and integration with C-RAM as an effective means of combating indirect fire.
\par
The system shall be deliverable in a self-contained package.  It shall be self-monitoring and self-reporting.  On initialization it shall provide lethal enforcement of a FOB without human assistance.  It shall, however, provide facility for operator override.  It shall not rely on FOB utility infrastructure.
\par The system shall operate in all weather and visibility conditions in which an enemy could emplou direct or indirect fires.\footnote{Course Project v2.1, ``Enclosure 1''}
\par It shall meet the following performance requirements: the operational availability (A$_{\rm O}$) shall be at least 90\% and the mean time between failures (MTBF) shall be no less than 750 hours.  Preventative and corrective maintenance is expected to follow a traditional O-level, I-level, and D-level paradigm.  Training shall emphasize CD- or web-based delivery.\footnote{ibid}
\par
The overall goal is 100\% effective base defense through security and deterrence by neutralization of suicide attacks, protection from direct and indirect fire, protection from enemy force invasion, and elimination of unlawful base penetration all while minimizing the impact on existing troop workload.  We shall provide a maximum effectiveness at minimal cost and risk to existing troops.
\par
The system shall be interoperable with existing systems such as the Army's C-RAM which may be considered a sub-system in the proposed design.  The system shall integrate with current policies and procedures regarding FOB security and shall in all cases support and augment current FOB security.  The system may also integrate with Command \& Control (C2) systems like FBCB2 Blue Force Tracker.
\par
Inputs shall include sensor data, some consumable lethality (e.g., bullets) and power.  Outputs shall include some consumable lethality (e.g., bullets).
\pagebreak
\begin{enumerate}
\item
Based on the analysis of preliminary system requirements it was determined that there is a need for additional security at Forward Operating Bases (FOB's) from those threats identified as direct and indirect fire, forceful and deceptive unlawful base penetration, and suicide bombings.  The threat is present and increasing with no abatement in sight.  This shall be developed as system to provide portable, effective, reliable, lethal force protection at small forward operating bases.  We can leverage existing highly effective systems like the Army's Counter-Rocket Artillery System (C-RAM).
\item
We need an automated way to make our FOB's more secure against enemy ground attacks.
\item
We need a system that picks up where C-RAM leaves off.  We shall focus on ground attacks and effective integration with C-RAM.
\item
The system shall meet the following performance requirements: the operational availability (A$_{\rm O}$) shall be at least 90\% and the mean time between failures (MTBF) shall be no less than 750 hours.  Preventative and corrective maintenance is expected to follow a traditional O-level, I-level, and D-level paradigm.  Training shall emphasize CD- or web-based delivery.
\par The system shall be deliverable in a self-contained package.  It shall be self-monitoring and self-reporting.  On initialization it shall provide lethal enforcement of a FOB without human assistance.  It shall, however, provide facility for operator override.  It shall not rely on FOB utility infrastructure.
\item
The threat is from direct and indirect fire, forceful and deceptive base penetration, and suicide bombings.  The threat is present and increasing with no abatement in sight.
\item
The overall goal is 100\% effective base defense though security and deterrence including:
\begin{itemize}
\item neutralization of suicide attacks
\item protection from direct and indirect fire
\item protection from enemy force invasion
\item elimination of unlawful base penetration
\item minimization to impact to existing force workload.
\end{itemize}
In short we shall provide maximum effectiveness at minimal cost and risk to existing troops.
\item
The system shall be interoperable with the Army's C-RAM which we may consider a sub-system of the proposed system.  It must also integrate with current policies and procedures regarding FOB security.  We may leverage existing C2 systems like FBCB2 BFT.
\item
Inputs: sensor data, some consumable lethality (e.g., bullets), and power. \\
Outputs: some consumable lethality (e.g., bullets).
\item
See above.
\end{enumerate}

\end{document}
