\documentclass[letterpaper,10pt]{article}

%\setlength{\parindent}{0in}
%\usepackage{fullpage} 
\usepackage{amsmath}
\usepackage{amssymb}
\usepackage{enumerate}
\usepackage{graphicx}
\usepackage[table]{xcolor}
\usepackage{dcolumn}
\oddsidemargin 0.0in
\textwidth 6.5in
\newcolumntype{.}{D{.}{.}{-1}}
\newcommand*{\myalign}[2]{\multicolumn{1}{#1}{#2}}

%opening
\title{Assignment 1}
\author{Steve Mazza}
\date{January 20, 2012}

\begin{document}
\maketitle

\section*{Methodology}
For this exercise I treated the three cases as discrete, each being an individual modification of the original spreadsheet.

\section*{Results}
The OMOE of the baseline case (original, unaltered spreadsheet) is the lowest at 0.589.  Case A came in third with an OMOE of 0.594.  Case C was second with an OMOE of 0.614.  And the best overall model as measured by OMOE was Case B at 0.749, a better than 27\% improvement over the original model.  The results are summarized below.

\begin{table}[htdp]
\begin{center}
\begin{tabular}{l.}
\hline
\myalign{c}{\textbf{Case}} & \myalign{c}{\textbf{OMOE}} \\
\hline\hline
Baseline & 0.589 \\
Case A & 0.594 \\
Case B & 0.749 \\
Case C & 0.614 \\
\hline
\end{tabular}
\end{center}
\end{table}


\section*{Conclusion}
If cost were not a consideration, Case B represents the best selection of the new SLC shop configuration.

It seems that the higher score of Case B is due to the relatively high value of the three affected attributes.  On the QFD1, \emph{Endurance} is 4$^{th}$ most heavily weighted.  And on QFD3, \emph{CIC}, \emph{Stores}, and \emph{Fuel Storage} (related to \emph{Endurance Range}) are the three most significant attributes.

\end{document}