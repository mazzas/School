\documentclass{beamer}


\usetheme{AnnArbor}
\usecolortheme{default}


\title{Applying Pattern Concepts to Systems (Enterprise) Architecture}
%\subtitle{Complex Systems Engineering}
\author{Steve Mazza}
\institute[Naval Postgraduate School]
{ 
    Naval Postgraduate School \\
    Monterey, CA \\
    \includegraphics[height=3cm]{images/NPS_logo.jpg}
}
\date {SE4920, Summer/2014}
\subject{Reverse Engineering}


\begin{document}

\frame{\titlepage}


\frame{{Introduction}
  \begin{block}{Intent}
    ``\dots to describe the research being performed to investigate the application of patterns to the practice of systems architecting.''
  \end{block}
  Goal of the enterprise architect: develop and implement a complex system using a methodical and repeatable approach.

  \begin{block}{Definition}
    ``A pattern is a model or facsimile of an actual thing or action, which provides some degree of representation (an abstraction) to enable the recreation of that entity over and over again.''
  \end{block}
}

\frame{{Frameworks}
  Considerations:
  \begin{itemize}
    \item strategic business goals
    \item business rules
    \item existing and legacy systems
    \item applicable technologies
  \end{itemize}

  Frameworks (contain many similarities)
  \begin{itemize}
    \item Zachman
    \item EA$^3$Cube
    \item DoDAF
  \end{itemize}
}

\frame{{A Short History Of Patterns}
  Christopher Alexander (Architect) pioneered the use and documentation of patterns. He was the first to recognize their value.

  Documentation of patterns consists of:
  \begin{itemize}
    \item Name: should be descriptive and represent the proposed solution
    \item Context: addresses the setting for the problem
    \item Problem: describes what the pattern addresses (the challenge or solution)
    \item Solution: describes the application of the pattern
  \end{itemize}

  His establishment of patterns tried to lower the cognitive burden of design ``by exploring large design spaces on behalf of the architect.''
  Making useful generalizations about reusable elements helps reduce the difficulty in thinking about architecting solutions.
} 

\frame{{History}
  \framesubtitle{Patterns In Information Technology}
  IBM established the following steps in using patterns to address eBusiness:
  \begin{enumerate}
    \item Develop a high-level business description
    \item Develop a solution overview diagram
    \item Identify business patterns
    \item Identify integration patterns
    \item Identify composite patterns
    \item Identify application patterns
    \item Integrate a package into a solution
    \item Identify run-time patterns
    \item Identify run-time and product mappings
  \end{enumerate}
}

\frame{{History}
  \framesubtitle{Patterns In Software Development}
  \begin{itemize}
    \item Introduced by the Gang Of Four (Gamma, Helm, Johnson, and Vlissides
    \item Considerably furthered by Martin Fowler
    \item Establish common vocabulary
    \item Reduce costly late cycle changes
    \item Weave parts of the overall software system architecture into a whole
  \end{itemize}
}

\frame{{Documenting Software Patterns}
  \begin{block}{Fowler (2003)}
    ``When people write patterns, they typically write them in some standardized format -- as befits a reference.  However, there's no agreement as to what sections are needed because every author has his or her own ideas\dots''
  \end{block}
}

\frame{{Discovering Patterns}
  Patterns are usually discovered through our brain's natural tendency to create abstractions on complexity.

  Abstractions tend to suggest more general solutions which, if sufficiently developed and documented, can lead to more broadly applicable solutions.
}

\frame{{Capturing Implicit Knowledge With Patterns}
  \begin{itemize}
    \item Implicit knowledge is only useful if it is shared usefully (in a manner that allows its application).
    \item Patterns offer a formal method to capture key aspects of that knowledge and facilitate its transfer and reuse.
    \item Patterns are only useful if they can be used by others.
    \item Formal means of documentation ensure accurate transfer of implicit knowledge.
  \end{itemize}
}

\frame{{Potential Benefits Of Architectural Patterns}
  Benefits include:
  \begin{itemize}
    \item Improved communications with stakeholders and design teams
    \item Application of sound architectural concepts and implementations.
    \item Reuse
    \item Reduced effort with regard to system testing, integration, and maintenance
    \item Improved development efficiency and productivity
    \item Reduced cost of documentation (example)
    \item Control of complexity through use of well known patterns
  \end{itemize}
}

\frame{{Architectural Pattern Research}
  Two issues applying patterns to systems architecture
  \begin{itemize}
    \item Architecture of a system requires a higher level of abstraction than that found in the software that may be part of the system.
    \item Patterns need to address interfaces to non-software parts in the pattern description.
  \end{itemize}
}

\frame{{Documenting Architecture Patterns}
  Minimal sections (just like Alexander):
  \begin{itemize}
    \item Name
    \item Context
    \item Problem
    \item Solution
  \end{itemize}

  Other useful sections:
  \begin{table}
    \centering
    \begin{tabular}{lll}
      Aliases & Resulting context & Known issues \\
      Email & Keywords & Forces \\
      Related patterns & Sketch & Rationale \\
      Example & Interfaces & Date documented \\
      Authors & References &
    \end{tabular}
  \end{table}
}

\frame{{A Proposed Pattern Hierarchy}
  System architecture patterns are broken into:
  \begin{enumerate}
    \item Structural patterns: physical part of the architecture
    \item System requirements patterns: format of a properly formed requirement
    \item Systems engineering activities patterns: indicate how the process of engineering activity is performed
    \item Systems engineering roles patterns: describe how the engineering role is performed
    \item Systems process patterns
  \end{enumerate}
}

\frame{{When Not To Use Architecture Patterns}
  Downfalls:
  \begin{itemize}
    \item Structural constraints inherent in patterns limits creativity
    \item Experts within their domain see patterns as having little use
  \end{itemize}

  When to avoid using patterns:
  \begin{itemize}
    \item New or unique requirements preclude the existence of a pattern
    \item Looking for a unique solution
    \item Technological innovation out-paces developed patterns
  \end{itemize}
  ``For designs that are proven and effective, and addressing problems common across multiple systems an domains, there should be a strong motivation to leverage the benefits that can accrue from the application of patterns.''
}

\frame{{Summary}
  \begin{itemize}
    \item Patterns are models or abstractions of reality
    \item Complexity exceeds mental capacity
    \item Patterns can exist at multiple levels
    \item Patterns are a powerful part of the architect's toolbox
    \item Patterns help solve difficult problems by leveraging existing knowledge
    \item Patterns help minimize the possibility that details will fall through the cracks
    \item Patterns expedite knowledge transfer
  \end{itemize}
}

\end{document}
