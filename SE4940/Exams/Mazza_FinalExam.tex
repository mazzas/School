\documentclass[letterpaper,10pt]{article}

%\setlength{\parindent}{0in}
%\usepackage{fullpage} 
\usepackage{amsmath}
\usepackage{amssymb}
\usepackage{enumerate}
\usepackage{graphicx}
\usepackage[table]{xcolor}
\usepackage{dcolumn}
\oddsidemargin 0.0in
\textwidth 6.5in
\newcolumntype{.}{D{.}{.}{-1}}
\newcommand*{\myalign}[2]{\multicolumn{1}{#1}{#2}}

%opening
\title{Final Exam}
\author{Steve Mazza}
%\date{December 9, 2011}

\begin{document}
\maketitle

\begin{description}
\item[Question 1:]
Given the differential equation $\dfrac{d^4y}{dx^4}+\dfrac{d^2y}{dx^2}=0\equiv y''''+Ay''$ derive the equivalent system of first-order ordinary differential equations.  Thsi is a fourth order differential equation.  What order is the system of equations?  Is the system linear or nonlinear?  What does such a system of first-order ordinary differential equations represent?

\item[Question 2:]
The maxwell-Bloch equations are a sophisticated model for a laser and describe the dynamics of the wlectric field $E$, the mean polarization of the atoms $P$, and the population inversion $D$:
\begin{align*}
\dot{E} &= (P-E) \\
\dot{P} &= \gamma_1(ED-P) \\
\dot{D} &= \gamma_2(\lambda+1-D-EP)
\end{align*}
where $\gamma_1$ and $\gamma_2$ are decay rates of the atomic polarization and population inversion, respectively, and $\lambda$ is a pumping energy parameter.  The parameter $\lambda$ may be positive, negative, or zero; all other parameters are positive.  In the simplest case, $P$ and $D$ relax rapidly to steady values, and hence may be eliminated as follows.
\begin{enumerate}
\item Assuming $\dot{D}\approx 0$\ $\dot{P}\approx 0$, express $P$ and $D$ in terms of $E$, and thereby derive a first-order equation for the evolution of $E$.
\item Find all the fixed points of $E$.
\item Draw the bifurcation diagram of $E^*$ versus $\lambda$.  Distinguish between stable and unstable branches.
\end{enumerate}

\item[Question 3:]
What is this an example of?  What features are represented?

\item[Question 4:]
For the Lorenz equations
\begin{align*}
\dot{x} &= \sigma(y-x) \\
\dot{y} &= rx-y-xz \\
\dot{z} &= xy-bz
\end{align*}
with $\sigma=10, r=28$, and $b=2.66666$, and initial condition $x=1.0+\delta, y=1.0$, and $z=10$, determine how long it takes the absolute error between the ``true $x$ solution'' $(\delta=0)$ to grow from $\delta$ to 0.1. Calculate for $\delta$ values of 0.01, $10^{-4}, 10^{-6}, 10^{-8}$, and $10^{-10}$.  What does this tell you abotu the predictability versus measurement error?  Can you estimate the Liapunov exponent?

\item[Question 5:]
Consider the iterated map given by
\[x_{n+1} = \left\{
  \begin{array}{lr}
    rx_n &  0\le x_n \le 0.5 \\
    f(1-x_n) &  0.5\le x_n \le 1
  \end{array}
\right.
\]
where $0<r<2$.  What properties do you expect to see in the orbit diagram?  Is there any condition that might cause different behavior?  The Liapunov exponent is $\lambda=\mbox{ln\ }r$.  What does this tell you about the behavior?

\item[Question 6:]In your own words and using no more than one paragraph, describe the difference between complex and complicated systems.  That is, in your own opinion what distinguishes the two?

\item[Question 7:]
How are fractals and complexity related?

\item[Question 8:]
Define what an adaptive agent-based model is and briefly describe its characteristics.

\item[Question 9:]
In an engineering system consisting of various parts and mechanisms, what kinds of diversity are most applicable to determining complexity?  How might that diversity be measured?

\item[Question 10:]
What approaches are likely to [be] part of any attempt to harness complexity in an inherently complex system?

\end{description}
\end{document}