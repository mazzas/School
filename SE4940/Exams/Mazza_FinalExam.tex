\documentclass[letterpaper,10pt]{article}

%\setlength{\parindent}{0in}
%\usepackage{fullpage} 
\usepackage{amsmath}
\usepackage{amssymb}
\usepackage{enumerate}
\usepackage{graphicx}
\usepackage[table]{xcolor}
\usepackage{dcolumn}
\oddsidemargin 0.0in
\textwidth 6.5in
\newcolumntype{.}{D{.}{.}{-1}}
\newcommand*{\myalign}[2]{\multicolumn{1}{#1}{#2}}

%opening
\title{Final Exam}
\author{Steve Mazza}
%\date{December 9, 2011}

\begin{document}
\maketitle

\begin{description}
\item[Question 1:]
Given the differential equation $\dfrac{d^4y}{dx^4}+\dfrac{d^2y}{dx^2}=0\equiv y''''+Ay''$ derive the equivalent system of first-order ordinary differential equations.  This is a fourth order differential equation.  What order is the system of equations?  Is the system linear or nonlinear?  What does such a system of first-order ordinary differential equations represent?

%TODO: Answer question. (mazzas) Fri Jun 13 18:20:42 2014

\item[Question 2:]
The Maxwell-Bloch equations are a sophisticated model for a laser and describe the dynamics of the electric field $E$, the mean polarization of the atoms $P$, and the population inversion $D$:
\begin{align*}
\dot{E} &= (P-E) \\
\dot{P} &= \gamma_1(ED-P) \\
\dot{D} &= \gamma_2(\lambda+1-D-EP)
\end{align*}
where $\gamma_1$ and $\gamma_2$ are decay rates of the atomic polarization and population inversion, respectively, and $\lambda$ is a pumping energy parameter.  The parameter $\lambda$ may be positive, negative, or zero; all other parameters are positive.  In the simplest case, $P$ and $D$ relax rapidly to steady values, and hence may be eliminated as follows.
\begin{enumerate}
\item Assuming $\dot{D}\approx 0$\ $\dot{P}\approx 0$, express $P$ and $D$ in terms of $E$, and thereby derive a first-order equation for the evolution of $E$.
\item Find all the fixed points of $E$.
\item Draw the bifurcation diagram of $E^*$ versus $\lambda$.  Distinguish between stable and unstable branches.
\end{enumerate}

This is Strogatz problem 3.3.2, page 82.

%TODO: Answer question. (mazzas) Fri Jun 13 18:21:02 2014

\item[Question 3:]
What is this an example of?  What features are represented?

See Strogatz, page 146.

%TODO: Answer question. (mazzas) Fri Jun 13 18:21:20 2014

\item[Question 4:]
For the Lorenz equations
\begin{align*}
\dot{x} &= \sigma(y-x) \\
\dot{y} &= rx-y-xz \\
\dot{z} &= xy-bz
\end{align*}
with $\sigma=10, r=28$, and $b=2.66666$, and initial condition $x=1.0+\delta, y=1.0$, and $z=10$, determine how long it takes the absolute error between the ``true $x$ solution'' $(\delta=0)$ to grow from $\delta$ to 0.1. Calculate for $\delta$ values of 0.01, $10^{-4}, 10^{-6}, 10^{-8}$, and $10^{-10}$.  What does this tell you about the predictability versus measurement error?  Can you estimate the Liapunov exponent? (Strogatz, 366)

This looks just like Strogatz, page 339.
%TODO: Answer question. (mazzas) Fri Jun 13 18:21:38 2014

\item[Question 5:]
Consider the iterated map given by
\[x_{n+1} = \left\{
  \begin{array}{lr}
    rx_n &  0\le x_n \le 0.5 \\
    f(1-x_n) &  0.5\le x_n \le 1
  \end{array}
\right.
\]
where $0<r<2$.  What properties do you expect to see in the orbit diagram?  Is there any condition that might cause different behavior?  The Liapunov exponent is $\lambda=\mbox{ln\ }r$. (Strogatz, 366)  What does this tell you about the behavior?

%TODO: Answer question. (mazzas) Fri Jun 13 18:21:59 2014

\item[Question 6:]
  In your own words and using no more than one paragraph, describe the difference between complex and complicated systems.  That is, in your own opinion what distinguishes the two?

Axelrod, page 15.

%TODO: Answer question. (mazzas) Fri Jun 13 18:22:14 2014

\item[Question 7:]
How are fractals and complexity related?

See Mitchell, page 103.  The answer is, ``though fractal dimension.''

%TODO: Answer question. (mazzas) Fri Jun 13 18:22:33 2014

\item[Question 8:]
Define what an adaptive agent-based model is and briefly describe its characteristics.

%TODO: Answer question. (mazzas) Fri Jun 13 18:22:45 2014

\item[Question 9:]
In an engineering system consisting of various parts and mechanisms, what kinds of diversity are most applicable to determining complexity?  How might that diversity be measured?

%TODO: Answer question. (mazzas) Fri Jun 13 18:22:57 2014

\item[Question 10:]
What approaches are likely to [be] part of any attempt to harness complexity in an inherently complex system?

In ``Harnessing Complexity,'' Axelrod \& Cohen present a framework for harnessing complexity, which the refer to as the Complex Adaptive Systems approach.  They describe various techniques of variation, interaction, and selection that the user of a system can leverage to affect or sway the outcome of a complex adaptive system.

The following ideas are summarized from the section titled, ``What a User of the Framework Can Do,'' in the Conclusion of their book.

\begin{description}
  \item[Variation] \ \\
    \begin{itemize}
      \item Arrange organizational routines to generate a good balance between exploration and exploitation.
      \item Link processes that generate extreme variation to processes that select with few mistakes in the attribution of credit.
    \end{itemize}
  \item[Interaction] \ \\
    \begin{itemize}
      \item Build networks of reciprocal interaction that foster trust and cooperation.
      \item Assess strategies in light of how their consequences can spread.
      \item Promote effective neighborhoods.
      \item Do not sow large failures when reaping small efficiencies.
    \end{itemize}
  \item[Selection] \ \\
    \begin{itemize}
      \item Use social activity to promote the growth and spread of valued criteria.
      \item Look for shorter-term, finer-grained measures of success that can usefully stand in for longer-run, broader goals.
    \end{itemize}
\end{description}

Detailed explanations of these approaches can be found on pages 155 -- 158.

\end{description}
\end{document}
