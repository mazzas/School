
\documentclass{beamer}


\usetheme{AnnArbor}
\usecolortheme{default}


\title{Cellular Automata}
%\subtitle{Complex Systems Engineering}
\author{Steve Mazza}
\institute[Naval Postgraduate School]
{ 
    Naval Postgraduate School \\
    Monterey, CA \\
    \includegraphics[height=3cm]{images/NPS_logo.jpg}
}
\date {SE4940, Spring/2014}
\subject{Complex Systems Engineering}


\begin{document}

\frame{\titlepage}


\frame{{Introduction}
	\begin{columns}[c]
    	\column{.5\textwidth}
		A cellular automaton is a collection of "colored" cells on a grid of specified shape that evolves through a number of discrete time steps according to a set of rules based on the states of neighboring cells. The rules are then applied iteratively for as many time steps as desired.
		
		\hfill--Wolfram MathWorld
	\column{.5\textwidth}
	\begin{figure}[h!]
		\begin{center}
	     		\includegraphics[width=0.5\textwidth]{images/GameOfLife.jpg}
     		\end{center}
		\caption{``Game Of Space'' on exhibit at the Museum of Contemporary Art in Hiroshima.}
	\end{figure}
    	\end{columns}
}

\frame{{Chapter 10}
	\framesubtitle{Cellular Automata, Life, and the Universe}
	%TODO
}

\frame{{Computation in Nature}
	%TODO
}

\frame{{Cellular Automata}
	%TODO
}

\frame{{The Game of Life}
	%TODO
}

\frame{{The Four Classes}
	\begin{description}
		\item[Class 1: ]
			Quickly settle to the same uniform final pattern independent of initial configuration.
		\item[Class 2: ]
			Produce either a uniform or cyclical patterns that are sensitive to the initial configuration.
		\item[Class 3: ]
			Produce mostly random behavior with some regular structures present.
		\item[Class 4: ]
			A mixture of order and randomness: simple localized structures are produced which interact with each other in complicated ways.
	\end{description}
}

\frame{{Woldfam's ``New Kind of Science''}
	Wolfram's proposed principle (in four parts):
	\begin{enumerate}
		\item The proper way to think about processes in nature is that they are \emph{computing}.
		\item Since even very simple rules can support universal computation, the ability to support universal computation is very common in nature.
		\item Universal computation is an upper limit on the complexity of conputations in nature.  That is, no natural system or process can produce behavior that is \emph{noncomputable}.
		\item The computations done by different processes in nature are almost always equivalent in sophistication.
	\end{enumerate}
}

\frame{{Chapter 11}
	\framesubtitle{Computing with Particles}
	%TODO
}

%TODO: Delete everything below here to the final frame.

\frame{{Block Types}
    \begin{block}{This is a Block}
        This is important information
    \end{block}
 
    \begin{alertblock}{This is an Alert block}
        This is an important alert
    \end{alertblock}
 
    \begin{exampleblock}{This is an Example block}
        This is an example 
    \end{exampleblock}
}

\frame{{Questions?}
	\begin{center}
		\includegraphics[width=.7\textwidth]{images/fin.png}
	\end{center}
}

\end{document}
