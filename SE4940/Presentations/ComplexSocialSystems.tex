\documentclass{beamer}


\usetheme{AnnArbor}
\usecolortheme{default}


\title{Models of Complex Adaptive Social Systems}
%\subtitle{Complex Systems Engineering}
\author{Steve Mazza}
\institute[Naval Postgraduate School]
{ 
    Naval Postgraduate School \\
    Monterey, CA \\
    \includegraphics[height=3cm]{images/NPS_logo.jpg}
}
\date {SE4940, Spring/2014}
\subject{Complex Systems Engineering}


\begin{document}

\frame{\titlepage}


\frame{{A Basic Framework}
  The elements of the eightfold path can be mapped to key modeling issues in complex social systems.
}

\frame{{The Eightfold Way}
  \begin{columns}[c]
    \column{0.4\textwidth}
      \begin{itemize}
        \item Wisdom
          \begin{itemize}
            \item Right View
            \item Right Intention
          \end{itemize}
        \item Ethical Conduct
          \begin{itemize}
            \item Right Speech
            \item Right Action
            \item Right Livelihood
          \end{itemize}
        \item Concentration
          \begin{itemize}
            \item Right Effort
            \item Right Mindfulness
            \item Right Concentration
          \end{itemize}
      \end{itemize}
    \column{0.6\textwidth}
    \begin{center}
      \includegraphics[scale=0.75]{images/noblecomix.jpg}
    \end{center}
  \end{columns}
}

\frame{{Smoke and Mirrors: the Forest Fire Model}
  %TODO: Enter reminder. (mazzas) Sun May 25 09:53:57 2014
}

\frame{{Eight Folding into One}
  %TODO: Enter reminder. (mazzas) Sun May 25 11:21:06 2014
}

\frame{{Conclusion}
  \begin{itemize}
    \item Different levels of adaptation impact behavior.  The emergence of firewalls in the forest fire model demonstrates how collective intelligence can arise.  Miller points to the interesting space between fixed rules and cognitive closure, referring to it as, ``both clever and messy.''
    \item Applying different labels allow us to use the same model in a variety of different domains across the same problem class (e.g., forest fires, bank failures, urban-suburban migration).
    \item ``The promise of uncovering deep connections among apparently disparate complex social systems is an important one.''
  \end{itemize}
}

\frame{{Complex Adaptive Social Systems in One Dimension}
  %TODO: Enter reminder. (mazzas) Sun May 25 09:40:40 2014
}

%TODO: Delete everything below here to the final frame.
\frame{{Example of columns 1}
    \begin{columns}[c]      % the "c" option specifies center vertical alignment
    \column{.5\textwidth}   % column designated by a command
     Contents of the first column
    \column{.5\textwidth}
     Contents split \\ into two lines
    \end{columns}
}
 
\frame{{Example of columns 2}
    \begin{columns}[t]      % contents are top vertically aligned
    \begin{column}[T]{5cm}  % each column can also be its own environment
        Contents of first column \\ split into two lines
    \end{column}
    \begin{column}[T]{5cm}  % alternative top-align that's better for graphics 
        \includegraphics[height=3cm]{images/NPS_logo.jpg}
    \end{column}
    \end{columns}
}

\frame{{Block Types}
    \begin{block}{This is a Block}
        This is important information
    \end{block}
 
    \begin{alertblock}{This is an Alert block}
        This is an important alert
    \end{alertblock}
 
    \begin{exampleblock}{This is an Example block}
        This is an example 
    \end{exampleblock}
}

\frame{{Questions?}
	\begin{center}
		\includegraphics[width=.7\textwidth]{images/fin.png}
	\end{center}
}

\end{document}
