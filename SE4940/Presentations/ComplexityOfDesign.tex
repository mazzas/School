\documentclass{beamer}


\usetheme{AnnArbor}
\usecolortheme{default}


\title{Understanding the Complexity of Design}
%\subtitle{Complex Systems Engineering}
\author{Steve Mazza}
\institute[Naval Postgraduate School]
{ 
    Naval Postgraduate School \\
    Monterey, CA \\
    \includegraphics[height=3cm]{images/NPS_logo.jpg}
}
\date {SE4940, Spring/2014}
\subject{Complex Systems Engineering}


\begin{document}

\frame{\titlepage}


\frame{{Introduction}
  %TODO: Enter reminder. (mazzas) Tue Jun 10 20:21:36 2014
}

\frame{{Targeted Approaches to Complexity in Design}
  \framesubtitle{Computational Complexity}
  This is almost a purely algorithmic approach which is often applied on computer science.
}

\frame{{Targeted Approaches to Complexity in Design}
  \framesubtitle{Complexity in Axiomatic Design}
  \begin{block}{Real Complexity}
    A measure of uncertainty in achieving the specified functional requirements.
  \end{block}
  \begin{block}{Imaginary Complexity}
    The uncertainty associated with a designer's lack of knowledge.
  \end{block}
}

\frame{{Targeted Approaches to Complexity in Design}
  \framesubtitle{Measuring Design Problem Complexity}
  \begin{block}{Underlying Assumption}
    ``The more coupled the design problem, the more complex it is.''
  \end{block}
  
  \begin{itemize}
    \item Look for interaction among design variables and targets.
    \item May be modelled by a series of linear equations where distance of the coefficients from the diagonal is an indicator of complexity.
  \end{itemize}
}

\frame{{Targeted Approaches to Complexity in Design}
  \framesubtitle{Measuring Artifact Complexity}
  \begin{block}{}
    Measuring the complexity of engineering artifacts may serve as a surrogate for problem or solution complexity.
  \end{block}

  In the case where a standard framework (or system) is used, artifact complexity can be gauged based on the effort required to document it.
}

\frame{{Rosen's Approach}
  \framesubtitle{Impredicativities in Science}
  \begin{block}{Formalism}
    In any apparently predictive formalism the application of certain larger contexts generated impredicativities that the original formalism cannot handle.
  \end{block}

  \begin{exampleblock}{Classic Example}
    This statement is false.
  \end{exampleblock}

  ``If the statement is true, then it is not false, but if it is not false, then it must be true, but it cannot be true because it says it is false, but if that is true, then it \emph{is} false, but, but, but\dots \emph{ad infinitum}.''
}

\frame{{Rosen's Approach}
  \framesubtitle{Implications for Technology and Design}
  \begin{block}{The Human Factor}
    ``In design, the semantic, non-rational, non-algorithmic, impredicative, subjective, and unpredictable nature of humanity is inescapable, because artifacts are always designed for human use, usually designed by humans themselves, and situated within a larger context of a complex world economy.''
  \end{block}
  In the author's view it is not surprising that design based on a Newtonian perspective have failed to achieve sufficient maturity.  The suggested alternative is to assume that design is complex.
}

\frame{{Understanding Design}
  \framesubtitle{History and Overview}
  The emergence of the science of complexity is largely credited to these realizations:
  \begin{enumerate}
    \item many interesting and unsolved problems are complex in nature,
    \item complexity spans a wide variety of problem domains, and
    \item complexity itself is an area worth studying.
  \end{enumerate}
  \begin{block}{Complex System}
    A large group of strongly interacting parts exhibiting nonlinear dynamical behavior.
  \end{block}
  May be classified as non-adaptive or adaptive.
}

\frame{{Understanding Design}
  \framesubtitle{Complex Adaptive Systems}
  Complex systems seem to operate in the following cycle:
  \begin{enumerate}
    \item coarse graining of information from the real world
    \item identification of perceived regularities
    \item compression into a schema
    \item variation of schema
    \item use of the schema
    \item selection pressures affecting competition
  \end{enumerate}
  The definition presented seems to indicate that complex adaptive systems violate the law of entropy.  However, it is pointed out that CAS are open systems, exchanging energy with their environment.  And some energy is used to change their internal state.  Lastly it is pointed out that entropy only applies to closed systems, which resolves the apparent contradiction.  CAS can often decrease entropy.
}

\frame{{Understanding Design}
  \framesubtitle{The Designer-Artifact-User Complex System}
  This model encompasses the three major subsystems of a \emph{design system}, 
  \begin{enumerate}
    \item the designer(s) of the artifact,
    \item the artifact(s) being designed, and
    \item the user(s) of the artifact.
  \end{enumerate}
  \begin{itemize}
    \item DAU system is situated in a larger environment
    \item Each subsystem (D-A-U) may not be singular
  \end{itemize}
}

\frame{{Understanding Design}
  \framesubtitle{DAU as a CAS}
  The authors establish the basis for accepting DAU as a complex adaptive system by considering each phase of the CAS cycle as it applies to DAU.
}

\frame{{Understanding Design}
  \framesubtitle{Coarse Graining Of Information from the Real World}
  \begin{block}{Characterization}
    \begin{enumerate}
      \item In a DAU system, this is the problem definition stage of design.
      \item It involves obtaining information from the real world.
      \item It requires the designer and user subsystems to interact with the environment.
    \end{enumerate}
  \end{block}
  \begin{exampleblock}{Problems}
    \begin{enumerate}
      \item \textbf{DAU:} Trade off between coarseness and fitness.
      \item \textbf{Design:} Trade off between spending time to understand the problem and delaying time to market.
    \end{enumerate}
  \end{exampleblock}
}
\frame{{Understanding Design}
  \framesubtitle{Identification Of Perceived Regularities}
  \begin{block}{Characterization}
    \begin{enumerate}
      \item Further refinement of the understanding of the problem
      \item May result in requirements with associated constraints, criteria, and goals.
      \item Primarily requires the designer and the users.
    \end{enumerate}
  \end{block}
  \begin{exampleblock}{Problems}
    \begin{enumerate}
      \item \textbf{CAS:} May mistake regularity for randomness.
      \item \textbf{Design:} Difficulty in interpreting user and other data that describes the problem.
    \end{enumerate}
  \end{exampleblock}
}

\frame{{Understanding Design}
  \framesubtitle{Compression Into Schema}
  \begin{block}{Characterization}
    \begin{enumerate}
      \item Equivalent to the conceptual design phase in a DAU system.
      \item Scope is narrowed in order to arrive at a solution space.
      \item Results in a full system solution.
      \item Involves the designer and artifact subsystems.
    \end{enumerate}
  \end{block}
  \begin{exampleblock}{Problems}
    \begin{enumerate}
      \item \textbf{CAS:} Difficulty estimating the continual evolution of the system.
      \item \textbf{Design:} Marketplace often moves quickly.
      \item \textbf{CAS:} Trade off between compression and time \& computation.
      \item \textbf{Design:} Trade off between the number of solution concepts and time \& money.
    \end{enumerate}
  \end{exampleblock}
}

\frame{{Understanding Design}
  \framesubtitle{Variation of Schema}
  \begin{block}{Characterization}
    \begin{enumerate}
      \item Designer improves, tests, and refines the concept.
      \item May require iteration of earlier phases.
      \item Equivalent to the variation of schemata phase in CAS.
      \item Primarily involves the designer and artifact subsystems.
    \end{enumerate}
  \end{block}
  \begin{exampleblock}{Problems}
    \begin{enumerate}
      \item \textbf{CAS:} Progress may be slow and methodical.
      \item \textbf{Design:} Unsure how to engineer revolutionary or disruptive change.
    \end{enumerate}
  \end{exampleblock}
}

\frame{{Understanding Design}
  \framesubtitle{Use Of the Schema}
  \begin{block}{Characterization}
    \begin{enumerate}
      \item The artifact is released onto the marker.
      \item May include a manufacturing process.
      \item Primarily involves the artifact and the user subsystems.
    \end{enumerate}
  \end{block}
  \begin{exampleblock}{Problems}
    \begin{enumerate}
      \item \textbf{CAS:} Difficulty incorporating new data.
      \item \textbf{Design:} Difficulty designing for the real world.
    \end{enumerate}
  \end{exampleblock}
}

\frame{{Understanding Design}
  \framesubtitle{Selection Pressure and Competition}
  \begin{block}{Characterization}
    \begin{enumerate}
      \item Free market forces at work.
      \item Involves designers, artifacts, and users.
    \end{enumerate}
  \end{block}
  \begin{exampleblock}{Problems}
    \begin{enumerate}
      \item \textbf{CAS:} ``Fitness is an elusive concept.''
      \item \textbf{Design:} Difficulty optimizing for constant change and imperfect environments.
      \item \textbf{CAS:} Poorly adapted schemas due to mismatched time scales.
      \item \textbf{Design:} Rapid or unanticipated market changes cause product failure.
    \end{enumerate}
  \end{exampleblock}
}

\frame{{Properties of Affordances}
  \framesubtitle{Relational Questions and Answers}
  Three questions to be answered:
  \begin{itemize}
    \item What is the nature of the relationship between users and artifacts?  What determines how an artifact may be used?
    \item What is the nature of the relationship between designers and artifacts?  Designers create the affordances of the artifacts.
    \item What is the nature of the relationship between designers and users?  Users inform designers of desired uses.
  \end{itemize}
  \begin{block}{Affordances}
    ``The affordances of the environment are what it offers the [user], what it provides or furnishes, either for good or ill.''
  \end{block}
}

\frame{{Properties of Affordances}
  \framesubtitle{Relational vs. Transformative Nature of Functions}
  \begin{exampleblock}{Transformative Paradigm}
    This is a highly algorithmic (tidy) environment which is often referred to at the standard mechanistic paradigm for design.
  \end{exampleblock}
  \begin{exampleblock}{Relational Paradigm}
    This is characterized by a high degree of coupling among designers, artifacts, and users.
  \end{exampleblock}
  Which idea is more critical hinges on the importance of interactions among designers, artifacts, and users.
  \begin{enumerate}
    \item When interactions are important, a relational affordance based paradigm is indicated.
    \item When interactions are less important, a transformative (or mechanistic) based paradigm is indicated.
  \end{enumerate}
}

\frame{{Summary}
  Due to its tightly coupled relational character, the concept of affordances more appropriately supports complexity in design than the transformative (mechanistic) paradigm.

  Affordances are a powerful tool to help understand the design process.
}

\frame{{Questions?}
	\begin{center}
		\includegraphics[width=.7\textwidth]{images/fin.png}
	\end{center}
}

\end{document}
