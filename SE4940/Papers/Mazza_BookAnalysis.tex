\documentclass[jou,apacite]{apa6}

\title{Book Analysis: Harnessing Complexity}
\shorttitle{Book Analysis}

\author{Steve Mazza}
\affiliation{Naval Postgraduate School}

\abstract{This is an example of a journal article using the \texttt{apa6.cls} document class to typeset manuscripts according to 6th edition of the Americal Psychological Association (APA) manual}

\rightheader{Book Analysis}
\leftheader{Steve Mazza}

\begin{document}
\maketitle    
                        
\section{Introduction}
Throughout this book the authors develop and present a framework for coming to terms with complexity that we are likely to find in systems with which we interact ever day.  Banking, the global commerce that results in affordable goods, the Internet, and our relationships at work are all examples of complex adaptive systems.

They begin with an introduction of the terminology and develop ideas connecting these terms to the real world.  They then introduce the mechanisms of variation, interaction, and selection, which are all treated in the context of the presented framework.  We will discuss these in the sections that follow.

The authors provide a summary of how the parts of the framework form a working whole.  While they present this in the conclusion, we find it a provides useful context during the rest of the discussion.
\begin{quote}
  Agents, of a variety of types, use their strengths in patterned interaction, with each other and with artifacts.  Performance measures on the resulting events drive the selection of agents and/or strategies through processes of error-prone copying and recombination, thus changing the frequencies of the types within the system.~\cite[page 154]{Axelrod}
\end{quote}
While this description may seem overly abstract, it is actually a succinct recitation of the framework and makes sense in the context of the ontology presented below.

\section{Background}

The following are the central ideas that have guided the development of the framework.  They are,
\begin{itemize}
  \item the difficulty of prediction in complex settings,
  \item how related themes of complexity have arisen in the physical, biological, and social sciences,
  \item how concepts from complexity studies can be useful in settings whose consequences are hard to predict, and
  \item the main mechanisms and design principles that we identify from complex systems research.
\end{itemize}

\subsection{Coming To Terms}

The authors introduce key terms and develop definitions for the following, which we will present and briefly comment on.  The identification of these key terms serves as the basis for effectively communicating the ideas about the framework. They create a meaningful ontology that we can use to extend the ideas presented in this book to the complex adaptive systems we encounter every day.

\subsubsection{Agent}
The agent is the component of the model responsible for purposeful action.  Agents interact with their environment in meaningful ways and, while frequently thought of as people, can be any entity capable of initiating action such as businesses, countries, or computer programs.  In many models, agents play a central and prominent role.  The agent is often the benefactor of change or optimization in the model and frequently (but not always) the motivation for creating the model in the first place.

\subsubsection{Strategy}
The strategy is the set of rules that an agent applies in response to changing environmental conditions.  In a model, these are the rules that we adjust in order to affect a desired outcome.  In real life, this is the set of heuristics and actions we apply in an attempt to have things go our way.  We may chose to cooperate or compete or to be patient or aggressive.  But we chose our strategy based on assumptions of achieving a desired outcome.  The authors make note of how strategies can
be adapted over time based on the achievement (or lack of achievement) of the agent's goals.

\subsubsection{Measure of Success}
How an agent sees the acquisition of or progress toward desired outcomes is gauged by measures of success.  These can be any metrics that the agent can observe within the environment.  They are often used to inform changes to strategy.  As an example, suppose you are trying to get some help on a project at work.  You have asked nicely several times to no good effect.  The measure of success is the level of assistance you have been able to get on the project.  If it is too low, you may alter
your strategy by involving a supervisor.

\subsubsection{Copying}
This refers to the duplication of effort by actors usually through the transfer of knowledge or information about how to affect an outcome.   This transfer of knowledge can be exact like in the case of a computer or error prone like in the case of the training of a newly hired employee.  Error in copying is sometimes beneficial because it introduces a randomness that adds to the diversity of the population, or group of agents.

\subsubsection{Population}
A population is a collection of agents or groups of agents and constitutes the entirety of those agents either as a whole or by some more narrowly defined characteristic.  It is also possible to think about populations of strategies.

\subsubsection{Type}
A type is a characteristic or attribute that defines a segment of the whole population.

\subsubsection{Variation}
This is the diversity among types in a population and often results from copying errors, as in evolution.  In many circumstances too much variety can lead to low copy rates (e.g., low birth rate).  For example if there are not a sufficient number of suitable mates then a particular population type may dwindle.  But more often, variation leads to much more interesting combinations and consequences.  After all, variety is the spice of life!

\subsubsection{Interaction Patterns}
These are the ways in which agents are likely to interact.  They are influenced by rules as well as other environmental factors.  Consider the coffee shop that you frequent on your way to work.  Your choice to stop there is shaped not only by the courteous staff and good coffee but also by the topography, roads, traffic lights, and other commuters between your home and work.  It is shaped by your desire for coffee, which drives this interaction pattern. And it is shaped by the fact that
everyone at work makes lousy coffee.

\subsubsection{Artifacts}
These are used by agents and have properties similar to agents.  They may also have ``affordances'' that evoke specific behaviors from agents. A guitar has a shape that beckons the user to pick it up, hold it, and strum the strings.  The form of the instrument encourages its use.

\subsubsection{System}
This refers to the entire environment including the agents, strategies, artifacts, and pertinent environmental factors.  Changes in policy or practice as well as brand new creations all require work at a system level.

\subsubsection{Complex}
The authors describe complexity best as, ``[A] system is complex when there are strong interactions among its elements, so that current events heavily influence the probabilities of many kinds of later events.''~\cite[page 7]{Axelrod}  Pay attention to the fact that complexity does not necessarily mean there are a lot of moving parts.  But that the parts that move have strong interactions.

\subsubsection{Selection}
This can be thought of as the way in which variety finds favor (or benefit).  Variation yields new combinations, sometimes by accident, occasionally resulting in a stronger or more capable agent.  If selection favors a particular agent then that agent will become more prolific.  The opposite is also true.

\subsubsection{Adaptation}
This is the result of a favorable selection process.  It can be thought of as marking an event that ratchets forward the progress of an agent toward some desirable end.

\subsubsection{Complex Adaptive System}
Any system with a population that seeks to adapt can be called a complex adaptive system, given that the system itself is sufficiently complex.

\subsubsection{Co-evolutionary Process}
This is the result of multiple populations of agents adapting to each other.  The idea is that any one population of agents would, itself, constitute a complete system, along with artifacts, strategies, and other appropriate environmental factors.

\subsubsection{Harnessing Complexity}
``[D]eliberately changing the structure of a system in order to increase some measure of performance, and to do so by exploiting an understanding that the system itself is complex.''~\cite[page 9]{Axelrod}

\subsubsection{Emergent Properties}
These are attributes of a system that are not individually accounted for by any of the constituent parts.  Emergence does not always occur.  But when it does, it imparts a value to the system beyond the component parts.

\subsubsection{Complicated}
Versus complex (see previous term), a system is complicated if it simply has many moving parts.

\subsubsection{Attribution of Credit}
This describes the process by which the winning and losing strategies are determined.

\subsubsection{Designer}
The introduction of new strategies and artifacts is done by the designer.

\subsubsection{Policy Makers}
By increasing rewards for some outcome or altering some pattern of interaction, policy makers alter the consequences of available strategies with malice of forethought.

The author introduces three fundamental questions to be developed over the course of the book.
\begin{enumerate}
  \item What is the right balance between variety and uniformity?
  \item What should interact with what, and when?
  \item Which agents or strategies should be copied and which should be destroyed?
\end{enumerate}

The following are the central concepts that do most of the work in the Complex Adaptive Systems approach presented by the authors.
\begin{itemize}
  \item Strategy, a conditional action pattern that indicates what to do in which circumstances.
  \item Artifact, a material resource that has definite location and can respond to the actions of agents.
  \item Agent, a collection of properties (especially location), strategies, and capabilities for interacting with artifacts and other agents.
  \item Population, a collection of agents, or, in some situations, collections of strategies.
  \item System, a larger collection, including one or more populations of agents and possibly also artifacts.
  \item Type, all the agents (or strategies) in a population that have some characteristic in common.
  \item Variety, the diversity of types within a population or system.
  \item Interaction pattern, the recurring regularities of contact among types within a system.
  \item Space (physical), the location in geographical space and time of agents and artifacts.
  \item Space (conceptual), the ``location'' in a set of categories structured so that ``nearby'' agents will tend to interact.
  \item Selection, processes that lead to an increase or decrease in the frequency of various types of agents or strategies.
  \item Success criteria or performance measure, a ``score'' used by an agent or designer in attrubuting credit in the selection of relatively successful (or unsuccessful) strategies or agents.
\end{itemize}

\section{Variation}

\section{Interaction}

\section{Selection}

\section{Related Work}

\section{Conclusion}

\bibliography{Mazza_BookAnalysis}

\end{document}
