\documentclass[jou,apacite]{apa6}

\title{Applying Recursion and Fractals As Design Patterns In Complex Systems}
\shorttitle{Term Paper}

\author{Steve Mazza}
\affiliation{Naval Postgraduate School}

\abstract{The term paper must be at least ten pages in length (single-spaced, normal font size, graphics are included in page count).  There is no maximum length, although a concise, well-crafted analysis is clearly superior to a lengthy, rambling narrative.  The subject of the term paper will be the student's synthesis of how some aspect of complexity, chaos, and other related topics relates to engineering of complex systems.  That is, how does each student think at least one of these topics can be integrated into a general approach to systems engineering.  The term paper will be due in class during Week 12.}

\rightheader{Term Paper}
\leftheader{Steve Mazza}

\begin{document}
\maketitle    
                        
\section{Introduction}  % Target: 1 page
Present what I am going to say.~\cite[page 112]{Mandelbrot}

I will show how applying recursive design patterns to complex systems reduces the complexity of their description.  Furthermore, patterns for the recursive descriptions can be found in some classes of fractals.  There exists an analog between certain fractals and the recursive descriptions for complex problems.

\section{Background}  % Target: 2 pages
Discuss descriptions of complex problems.  Introduce methods of simplifying complexity.  Introduce recursion as a solution to some classes of problems.
\subsection{Definition of Complexity}  % Target: 1/2 page.
\begin{enumerate}
  \item Introduce the term
  \item Provide (and cite) a definition
  \item Discuss interactions and how they lead to complexity
  \item Provide an example showing the difference between complex and non-complex systems.
\end{enumerate}

\subsection{How to Describe Complex Systems}  % Target: 1/2 page.
\begin{enumerate}
  \item Discuss the importance of formal methods.
  \item Talk about frameworks and discuss some standards.
  \item Briefly talk about design patterns.
  \item Talk about abstraction and simplification.
\end{enumerate}

\subsection{Methods Of Simplifying Complexity}  % Target: 1/2 page.
\begin{enumerate}
  \item Introduce abstraction and discuss (definition and examples).
  \item Introduce encapsulation and discuss (definition and examples).
  \item Discuss interface definition and contracts.
  \item Discuss black box architecture.
\end{enumerate}

\subsection{Introducing Recursion}  % Target: 1/2 page.
\begin{enumerate}
  \item Provide (and cite) a definition.
  \item Introduce the concept through example.  Consider the description of some algorithm from computer science such as \emph{factorial}.
  \item Introduce the idea that recursion can be applied to the physical world and provide an example.
\end{enumerate}

\section{Fractals}  % Target: 3 pages
\subsection{Introduction}
Introduce several classes of fractals (with figures).
\begin{enumerate}
  \item Koch snowflake
  \item Sieve 
\end{enumerate}
Consult Benoit Mandelbrot for examples and explanation.

\subsection{Constructing Fractals}
Describe and demonstrate their construction, emphasizing self-similarity and recursion.

\subsection{Self-similarity and Fractals}
Discuss self-similarity and its connection to fractals.  Discuss how we can describe fractals at all levels of magnification by describing their construction at one given level.  Show how simple rules can lead to complexity.

\section{Design Patterns}  % Target: 3 pages
Introduce the notion of using design patterns to describe solutions to problems.  Point out that traditional patterns don't adequately encode recursion.  Propose the use of some classes of fractals in describing recursive solutions to complex problems.
\subsection{Use Of Design Patterns}
Introduce the history and use of design patterns.

\subsection{Shortcomings Of Traditional Design Patterns}
Show the shortcomings of traditional design patterns in modeling complex systems.

\subsection{Use of Fractals as Design Patterns}
Propose the use of some classes of fractals as design patterns.

\subsection{Example}
Work through an example (with figures)

\subsection{From Simplicity, Complexity}
Discuss how simple rules can quickly lead to complexity.

\section{Conclusion}  % Target: 1 page
Wrap up.  Summarize what was covered.  Restate the proposal to use certain classes of fractals to describe recursive solutions to complex problems.
\begin{enumerate}
  \item Summarize and wrap up.
  \item Restate the proposal
\end{enumerate}

\bibliography{Mazza_TermPaper}

\end{document}
