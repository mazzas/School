\documentclass[jou,apacite]{apa6}

\title{Using Recursion To Understand Complexity: A Systems Engineer's Approach to Fractals}
\shorttitle{Term Paper}

\author{Steve Mazza}
\affiliation{Naval Postgraduate School}

\abstract{The term paper must be at least ten pages in length (single-spaced, normal font size, graphics are included in page count).  There is no maximum length, although a concise, well-crafted analysis is clearly superior to a lengthy, rambling narrative.  The subject of the term paper will be the student's synthesis of how some aspect of complexity, chaos, and other related topics relates to engineering of complex systems.  That is, how does each student think at least one of these topics can be integrated into a general approach to systems engineering.  The term paper will be due in class during Week 12.}

\rightheader{Term Paper}
\leftheader{Steve Mazza}

\begin{document}
\maketitle    
                        
\section{Introduction}
Lorem ipsum dolor sit amet, consetetur sadipscing elitr, sed diam nonumy eirmod
tempor invidunt ut labore et dolore magna aliquyam erat, sed diam voluptua. At
vero eos et accusam et justo duo dolores et ea rebum. Stet clita kasd gubergren,
no sea takimata sanctus est Lorem ipsum dolor sit amet.~\cite[page 112]{Axelrod}

\section{Engineering Concerns}
\subsection{Measurement}
Talk about how things are measured at different scales.  Bring this discussion to the Koch Snowflake.  Explain the snowflake in detail and show lots of figures.  Tie it to differences in measurement, accuracy, significant digits, etc.  Show how this is a systems engineering concern.

\section{Conclusion}
Lorem ipsum dolor sit amet, consetetur sadipscing elitr, sed diam nonumy eirmod
tempor invidunt ut labore et dolore magna aliquyam erat, sed diam voluptua. At
vero eos et accusam et justo duo dolores et ea rebum. Stet clita kasd gubergren,
no sea takimata sanctus est Lorem ipsum dolor sit amet.

\bibliography{Mazza_TermPaper}

\end{document}
