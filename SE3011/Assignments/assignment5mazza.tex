\documentclass[letterpaper,10pt]{article}

%\setlength{\parindent}{0in}
%\usepackage{fullpage} 
\usepackage{amsmath}
\usepackage{amssymb}
\usepackage{enumerate}
\usepackage{graphicx}
\usepackage{dcolumn}
\oddsidemargin 0.0in
\textwidth 6.5in
\newcolumntype{.}{D{.}{.}{-1}}

%opening
\title{Assignment 5}
\author{Steve Mazza}
\date{November 23, 2011}

\begin{document}
\maketitle

\begin{enumerate}
	\item Using the USC COCOMO Suite along with the accompanying spreadsheet I determine the grand total cost of labor to be \$4,806,237.00.  The values calculated using COCOMO are expanded in the spreadsheet and total to \$4,609,613.00 due to rounding errors.  See the spreadsheet for explanation and monthly totals.
	\item Savings on the \emph{development} effort is a one-shot calculation.  Using the USC COCOMO model to first determine a baseline with \emph{Nominal} inputs yields the following:
	\begin{center}
		\begin{tabular}{l.}
			\hline \\
			Effort & 465.3 \\
			Schedule & 27.9 \\
			Cost & \$3,722,522.00 \\
			\hline
		\end{tabular}
	\end{center}
	Then I run the calculation again with the adjusted values for \emph{Use of Software Tools} and \emph{Language and Toolset Experience}, the former set to \emph{High} and the latter set to \emph{Low} as prescribed.
	\begin{center}
		\begin{tabular}{l.}
			\hline \\
			Effort & 465.5 \\
			Schedule & 27.7 \\
			Cost & \$3,651,794.00 \\
			\hline
		\end{tabular}
	\end{center}
	From these I determine the one-shot savings based on the development effort to be \$70,728.00.
	\par
	Then, using the numbers supplied in the lecture as the basis of calculation, I determine a baseline annual \emph{maintenance} effort as follows:
	\begin{center}
		\begin{tabular}{l.}
			\hline \\
			Effort & 79.3 \\
			Schedule & 15.5 \\
			Cost & \$634,132.00 \\
			\hline
		\end{tabular}
	\end{center}
	Then I run the calculation again with the adjusted values for \emph{Use of Software Tools} and \emph{Language and Toolset Experience}, the former set to \emph{High} and the latter set to \emph{Nominal} as prescribed.
	\begin{center}
		\begin{tabular}{l.}
			\hline \\
			Effort & 71.3 \\
			Schedule & 15 \\
			Cost & \$570,719.00 \\
			\hline
		\end{tabular}
	\end{center}
	This represents an annual savings of \$63,413.  Following the example in the lecture I multiply this savings over five (5) years and get an aggregate savings of \$317,065.00 based on maintenance.
	\par
	The total savings considering \emph{development} and \emph{maintenance} is \$387,793.00.  Taking into account the values supplied for investment cost (\$250,000.00) and total acquisition costs (\$4,400,000.00) ROI is calculated as follows:
	\begin{align*}
	ROI &= (387793 - 250000) / 4400000 \\
	&= 137793 / 4400000 \\
	&= 0.03
	\end{align*}
	This represents a little better than 3.1\% savings over five (5) years and so I find the investment worthwhile.
	\par
	Note: while the problem suggests a calculation of savings based on 4.4M for development and 8.8M for maintenance, we do not appear to have access to \emph{maintenance effort multipliers} which appear to differ from the \emph{development effort multipliers}, hence my use of COCOMO and the calculation above.  This may be slightly unorthodox but I believe is justified in that the savings attributable to maintenance is calculated as the difference in the maintenance costs which inherently takes those costs into account.
	\par
	Using an alternate calculation similar to that on slide \#9 of our lecture I determine the 5-year ROI to be approximately 5\% and so I still find the investment worthwhile.\footnote{See the accompanying spreadsheet for calculations and explanation.}
\end {enumerate}
\end{document}