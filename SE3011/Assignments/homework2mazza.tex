\documentclass[letterpaper,10pt]{article}

%\setlength{\parindent}{0in}
%\usepackage{fullpage} 
\usepackage{amsmath}
\usepackage{amssymb}
\usepackage{enumerate}
\usepackage{graphicx}
\usepackage{color}
\usepackage{dcolumn}
\oddsidemargin 0.0in
\textwidth 6.5in
\newcolumntype{.}{D{.}{.}{-1}}

%opening
\title{Homework 2}
\author{Steve Mazza}
\date{October 26, 2011}

\begin{document}
\maketitle

\begin{enumerate}
\item The conversion factor from capabilities to requirements is $1/0.077059$ or 12.9702 as demonstrated by the following model:
\begin{verbatim}
> fit1 <- lm(capabilities ~ requirements, data=dataFile1)
> summary(fit1)

Call:
lm(formula = capabilities ~ requirements, data = dataFile1)

Residuals:
    Min      1Q  Median      3Q     Max 
-2.5803 -1.3406 -0.6886  1.0469  4.2185 

Coefficients:
             Estimate Std. Error t value Pr(>|t|)    
(Intercept)  0.916200   0.957642   0.957    0.351    
requirements 0.077059   0.005099  15.112 1.14e-11 ***
---
Signif. codes:  0 �***� 0.001 �**� 0.01 �*� 0.05 �.� 0.1 � � 1 

Residual standard error: 2.089 on 18 degrees of freedom
Multiple R-squared: 0.9269,	Adjusted R-squared: 0.9229 
F-statistic: 228.4 on 1 and 18 DF,  p-value: 1.138e-11 
\end{verbatim}
Using the CER given to derive an 80\% confidence interval estimate of effort for 22 capabilities with my conversion factor:
\begin{center}
\begin{align*}
\mbox{effort(person months)} &= 38.55 \times (12.902\times 22)^{1.06} \\
&= 38.55 \times 285.3840^{1.06} \\
&= 38.55\times 400.6410 \\
&= 15444.71
\end{align*}
\end{center}
\item The covariance matrix and model parameters for the three CER's are:
\begin{verbatim}
> vcov(fit2)
            (Intercept)      Weight     DataRate
(Intercept) 14125099.33 -94781.4510 -13233.38143
Weight        -94781.45   1745.7161   -105.60854
DataRate      -13233.38   -105.6085     55.13398

> vcov(fit2a)
            (Intercept)     Weight
(Intercept)  73320342.0 -804468.66
Weight        -804468.7   10335.79

> vcov(fit2b)
            (Intercept)     DataRate
(Intercept) 12281183.22 -25942.62991
DataRate      -25942.63     66.67152

\end{verbatim}
The ANOVA table for the three models is given as:
\begin{verbatim}
> anova(fit2, fit2a, fit2b)
Analysis of Variance Table

Model 1: Cost ~ Weight + DataRate
Model 2: Cost ~ Weight
Model 3: Cost ~ DataRate
  Res.Df        RSS Df   Sum of Sq      F    Pr(>F)    
1     15  431645574                                    
2     16 3083288942 -1 -2651643368 92.147 8.535e-08 ***
3     16  629745888  0  2453543054                     
---
Signif. codes:  0 �***� 0.001 �**� 0.01 �*� 0.05 �.� 0.1 � � 1
\end{verbatim}
Based on my analysis I would say that Data Rate is a much better predictor for Cost than Weight.
\end {enumerate}
\end{document}