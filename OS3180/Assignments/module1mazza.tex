\documentclass[letterpaper,10pt]{article}

%\setlength{\parindent}{0in}
%\usepackage{fullpage} 
\usepackage{amsmath}
\usepackage{enumerate}
\oddsidemargin 0.0in
\textwidth 6.5in

%opening
\title{Homework for Module 1}
\author{Steve Mazza}

\begin{document}

\maketitle

\begin{description}
\item[1.1.10] 
The probability values of the three outcomes are as follows:
\begin{align*}
P(I) &= 0.6 \\ 
P(II) &= 0.3 \\ 
P(III) &= 0.1
\end{align*}

\item[1.2.6]
The probability that the red die will have a score that is \emph{strictly greater} than the blue die is $15/36$.  The probability is $< 0.05$ because of the requirement that the value of the second die be \emph{strictly greater} than value of the first.  The compliment of this event is $21/36$.

\item[1.3.2]
The probabilities of the events are as follows:
\begin{align*}
P(B) &= 0.08 + 0.13 + 0.06 + 0.01 + 0.11 + 0.05 + 0.02 \\
&= 0.46 \\
P(B\cap C) &=  0.02 + 0.05 + 0.11 \\
&= 0.18 \\
P(A \cup C) &= 0.07 + 0.05 + 0.01 + 0.02 + 0.05 + 0.08 + 0.04 + 0.11 + 0.11 + 0.07 \\
&= 0.61 \\
P(A \cap B \cap C) &= 0.02 + 0.05 \\
&= 0.07
\end{align*}
	
\item[1.4.14]
For the following $S =$ size, $T =$ taste, and $A =$ appearance.
\begin{enumerate}[a)]
\item 
\begin{align*}
P(S\mid T) &= \frac{P(S\cap T)}{P(T)} \\
&= \frac{0.69}{0.78} \\
&= 0.88
\end{align*}
\item 
From what was given in the text we can derive that
\begin{align*}
0.84 &= 1 - P(S'\cap A') \\
0.16 &= P(S'\cap A')
\end{align*}
And substituting back into the original equation we get
\begin{align*}
P(T\mid (S'\cap A')) &= P(0.78\mid 0.16) \\
&= \frac{P(T\cap S'\cap A')}{P(S'\cap A')} \\
&= \frac{0.04}{0.16} \\
&= 0.25
\end{align*}
\end{enumerate}

\item[1.5.7]
For clarity we will refer to switch \#1 as $S_{1}$, switch \#2 as $S_{2}$, and switch \#3 as $S_{3}$. \\ Calculating the combined probability for $S_{1}$ and $S_{2}$ we get $$S_{1+2} = 0.81$$ and we can infer the following
\begin{align*}
S_{1+2}' &= 0.19 \\
S_{3}' &= 0.1
\end{align*}
and so 
\begin{align*}
P(S_{1+2}\cup S_{3}) &= P(S_{1+2}'\cap S_{3}') \\
&= 1 - (0.19\times 0.1) \\
&= 0.98
\end{align*}

\item[1.5.8]
The general formula for determining the probability of a concurrent birthday ($P_{CB}$) given $n$ people is
$$P_{CB} = 1 - \frac{365!}{365^{n}(365 - n)}$$
The following values were obtained using this formula:
\begin{align*}
n &= 10 \\
&= 0.116 \\
n &= 15 \\
&= 0.252 \\
n &= 20 \\
&= 0.411 \\
n &= 25 \\
&= 0.569 \\
n &= 30 \\
&= 0.706 \\
n &= 35 \\
&= 0.814
\end{align*}
Twenty-three (23) people is the smallest value for which the probability is larger than a half.
\par Given that all dates are equally likely (i.e., ignoring February 29) I believe that birthdays are equally likely to be on any day of the year.

\item[1.6.4]
For the following exercise species 1 $= S_{1}$, species 2 $= S_{2}$, species 3 $= S_{3}$, and $T$ is the probability any given species is tagged.
\begin{align*}
P(S_{1}\mid T) &= \frac{P(S_{1}) P(T_{1}\mid S_{1})}{P(T)} \\
&= \frac{0.45\times 0.1}{0.187} \\
&= 0.24 \\
P(S_{3}\mid T) &= \frac{P(S_{2}) P(T_{2}\mid S_{2})}{P(T)} \\
&= \frac{0.38\times }{0.187} \\
&= 0.30 \\
P(S_{2}\mid T) &= \frac{P(S_{3}) P(T_{3}\mid S_{3})}{P(T)} \\
&= \frac{0.17\times 0.5}{0.187} \\
&= 0.45
\end{align*}

\item[1.7.4]
$$(5+3)\times 7\times 6\times8 = 2688$$

\item[1.7.10] \ 
\begin{enumerate}[a)]
\item
\begin{align*}
C_{52,5}  &= \frac{52!}{5!\times 47!} \\
&= 2598960
\end{align*}
\item
\begin{align*}
C_{13,5}  &= \frac{13!}{5!\times 8!} \\
&= 1287
\end{align*}
\end{enumerate}

\item[1.7.12]
\begin{align*}
P_{6,6} &= \frac{6!}{(6-6)!} \\
&= 720
\end{align*}

\item[1.9.33]
For the following exercise a lit warning light will be represented by $W$ and a fault will be represented by $F$.
\begin{align*}
P(F\mid W) &= \frac{P(F) P(W\mid F)}{P(F) P(W\mid F) + P(F') P(W\mid F')} \\
&= \frac{0.004\times 0.992}{(0.004\times 0.992) + (0.996\times 0.003)} \\
&= 0.57
\end{align*}

\end {description}

\end{document}
