\documentclass[letterpaper,12pt]{article}

%\setlength{\parindent}{0in}
%\usepackage{fullpage} 
\usepackage{amsmath}
\usepackage{amssymb}
\usepackage{enumerate}
\usepackage{graphicx}
\usepackage[table]{xcolor}
\usepackage{dcolumn}
\oddsidemargin 0.0in
\textwidth 6.5in
\newcolumntype{.}{D{.}{.}{-1}}
\newcommand*{\myalign}[2]{\multicolumn{1}{#1}{#2}}

%opening
\title{Modle 1 Homework}
\author{Steve Mazza}
\date{April 13, 2012}

\begin{document}
\maketitle

% Practice uploading an assignment.� Write a brief response to this question:� What are the key common aspects of the definitions of architecture? (using MS Word or equivalent).� Just a few lines, please!� Turn it in via the "Homework for Module 1" link in the Assignments area on Sakai.� Be sure to include your own last name and "Module 1" in the file name (ie, "Smith_Module_1.doc").

With the conspicuous exception of the \emph{IEEE Architecture Working Group (AWG)} every definition uses the term \emph{structure} and defines architecture in a way which is sufficiently abstract as to apply to software, hardware systems, or buildings.  Inasmuch, it is notable that the \emph{IEEE AWG} goes to such great lengths to avoid using the term \emph{structure}, a move intended to distance the definition from the traditional physical applications.

My unsolicited opinion is that this is a somewhat unnecessary move.  The analogy to the physical world applied to the architecture of the abstract provides a concrete (pun intended) perspective which is often useful in understanding the phases, roles, and principles that guide the architectural process.

Also common to most of the definitions is the notion that the architecture of a system is understood by its constituent parts.  The definition posited by our text says:
\begin{quote}
\textbf{Architecture:}  The structure -- in terms of components, connections, and constraints -- of a product, process, or element.
\end{quote}
The \emph{IEEE} definition references, ``levels of hierarchy.''  And the \emph{INCOSE SAWG} specifically says, ``\dots defined in terms of system elements\dots''  Likewise, \emph{MIL-STD-498} states:
\begin{quote}
\textbf{Architecture:}  The organizational structure of a system or CSCI, identifying its components, their interfaces, and a concept of execution among them.
\end{quote}
An abstract definition which is equally applicable to physical as well as logical systems and which is definable by its aggregate parts captures much of the key common aspects of the definitions of architecture discussed in Appendix C of \emph{The Art of Systems Architecting} (Maier \& Rechtin, 2009).

\end{document}