\documentclass[letterpaper,12pt]{article}

%\setlength{\parindent}{0in}
%\usepackage{fullpage} 
\usepackage{amsmath}
\usepackage{amssymb}
\usepackage{enumerate}
\usepackage{graphicx}
\usepackage[table]{xcolor}
\usepackage{dcolumn}
\oddsidemargin 0.0in
\textwidth 6.5in
\newcolumntype{.}{D{.}{.}{-1}}
\newcommand*{\myalign}[2]{\multicolumn{1}{#1}{#2}}


\title{Modle 2 Homework}
\author{Steve Mazza}
\date{April 20, 2012}

\begin{document}
\maketitle

% By now each project team should have selected one of the four systems available in the Class Project area of Sakai.  The purpose of the assignment this week is to get each and every student familiarizing themselves with project source material available to the project teams.  This is an individual assignment; by that I mean each and every student will submit a response and each and every student will receive an individual grade.  

% Accordingly, each and every student in this class should individually consider the role of the architect in the context of the system the team selected for the class project.  By "class project" I mean JCREW, CANES, et al depending on which project your team selected.  Each and every student shall individually identify who the system architect is for the project their team selected and how that role was defined.  

% I do NOT mean who is the Lead Architect for your team...I want you to tell me who the Lead Architect is for the system your team selected.  If a single "lead architect" cannot be identified explicitly, use the person or organization who took on that role - a de facto architect if you will.  Turn it in to me via the "Homework for Module 2" link in the Assignments area of Sakai.  Be sure to include your own last name and "Module 2" in the file name (ie, "Smith_Module_2.doc").

The architect for the US Naval program Automated Digital Network System Increment III (ADNS) is identified in the ADNS Increment III CDD V3.3 as SPAWAR System Center San Diego with development and training support from the Program Executive Office (PEO) C4I and Space, Program Manager, Warfare 160 (PMW-160).\footnote{US Department of the Navy, \emph{Capability Development Document for Automated Digital Network System (ADNS)}, 2005, pp 56.}

It is the architect's role to advocate for the client by capturing and transferring to the engineers the essence of what the client is willing to pay for.  First drawing from experience to understand, often through inference, the problem space that the client is trying to address and then, second, working with the client to understand the constraints under which a successful solution can be built and delivered.  After that initial information is captured it is conveyed to the engineers using domain-specific language that removes ambiguity.  The architect may develop a ssytem-of-systems view into which the desired product may fit.  In such a case the architect may help identify control points, critical information flow, and interfaces.  The architect remains on the project as the client advocate throughout its entire lifecycle assisting and facilitating the \emph{what} (but never the \emph{how}) that will be built, delivered, used, and eventually retired.

\end{document}