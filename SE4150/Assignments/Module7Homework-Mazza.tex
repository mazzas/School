\documentclass[letterpaper,12pt]{article}

%\setlength{\parindent}{0in}
%\usepackage{fullpage} 
\usepackage{amsmath}
\usepackage{amssymb}
\usepackage{enumerate}
\usepackage{graphicx}
\usepackage[table]{xcolor}
\usepackage{dcolumn}
\oddsidemargin 0.0in
\textwidth 6.5in
\newcolumntype{.}{D{.}{.}{-1}}
\newcommand*{\myalign}[2]{\multicolumn{1}{#1}{#2}}

\title{Module 7 Homework}
\author{Steve Mazza}
\date{June 1, 2012}

\begin{document}
\maketitle

% Now that you have made some progress in the class project, you should have some familiarity with CORE (either as a user, or having looked at the website and available supporting documentation)  in supporting your architecting efforts.  Comment on the efficacy of that tool's functionality in doing so.  List 3 aspects of CORE you like based on your reading and/or experience as a user and a few sentences on why you listed those.  List 3 aspects of CORE you dislike based on your reading and/or experience as a user and a few sentences on why you listed those.  One or two pages at the most. 

% Introduction
CORE is a mature and capable product that facilitates communication, design revision, and requirements traceability.  The cost for these benefits is principally in training and total cost of ownership but also requires the use of  Department of Defense Architecture Framework (DoDAF) to fully realize its potential.

% 3 likes
There is a lot that I like about CORE and so it is difficult to limit the discussion to just three aspects.  In general, though, CORE facilitates communication, automatically propagates design revisions, and maintains traceability back to requirements.  First, CORE facilitates communication by producing as an artifact of its use standard DoDAF views.  This is the standard documentation by which DoD projects are managed.  Adopting this standard ensures interoperability with other programs as well as acceptance into the DoD community.  Second, design revisions are automatically propagated throughout the CORE system which reduces the burden and overhead on the user to update all of the applicable views.  The automated design revision propagation also ensures an internal consistency that increases design integrity.  Lastly, CORE maintains a careful internal audit trail of all work products all the way back to the requirements, providing requirements traceability that is so important to non-trivial engineering projects.

% 3 dislikes
CORE is not without its drawbacks. It is a large and heavy framework whose mastery requires considerable training.  Leveraging its full power implies an investment in becoming proficient in its use, a task requiring considerable time and effort.  Further it assumes the use of DoDAF to fully leverage its power.\footnote{CORE Spectrum is required for DoDAF 2.0 and SysML support.}  An organization that standardizes on a different architecture framework or methodology will find CORE a less optimal solution to their needs.  Last, the cost of implementing CORE at the enterprise level is substantial.  The licensing costs, alone, are high and depending on how it is to be deployed there are considerable infrastructure and administrative requirements.  The total cost of ownership and deployment of CORE mandates a commitment to training, support, and use.

% Summary
While CORE is a mature and capable product that facilitates communication, design revision, and requirements traceability, it requires a commitment to training, the use of DoDAF, and substantial support to fully realize its potential.

\end{document}