\documentclass[letterpaper,10pt]{article}

%\setlength{\parindent}{0in}
%\usepackage{fullpage} 
\usepackage{amsmath}
\usepackage{amssymb}
\usepackage{enumerate}
\usepackage{graphicx}
\usepackage[table]{xcolor}
\usepackage{dcolumn}
\oddsidemargin 0.0in
\textwidth 6.5in
\newcolumntype{.}{D{.}{.}{-1}}
\newcommand*{\myalign}[2]{\multicolumn{1}{#1}{#2}}

%opening
\title{Evolution and Military Adoption of Commercial Smart Phones}
\author{Steve Mazza}
\date{September 3, 2012}

\begin{document}
\maketitle

% Make 2 parts per section: first about the technology in general and second about the Army's use of it.
\section*{Technology Description}
% Describe the technology and include pictures, drawings, or other illustrations as appropriate.
Smart phones are the result of the intersection of several mature and emerging consumer technologies including <list technologies>.  The smart phone revolution started in <name the date> with the introduction of the BlackBerry <model?>.  Fueled by consumer demand, Microsoft and Palm entered the market.  But it wasn't until Apple introduced the iPhone that the revolution fully took hold.  No longer just the tools of the business elite, now these devices were actively marketed to the average consumer -- equally at home in the boardroom and in your kid's backpack.

\section*{Technological Need}
% What are the needs the technology fulfills � think back to SE3100 when you did a capability needs analysis.  Define and/or model as appropriate the needs (wants) the technology fulfills.  What functions does the technology provide?  How does it fit into its environment?
More and more, our lives put us on the go.  In a consumer driven market, manufacturers respond to the needs of our lifestyles.  Whether for work or personal use, our smart phones span the distance from home, to the gym, to work, and are even with us on our commute time in between.  In the Army's case, warfighters' needs are not all that different and smart phones are finding a niche in stateside training commands, permanent CONUS duty stations, and at different echelons in theater from Division all the way down to the dismounted soldier on patrol.  

Key enablers of these devices both for the Army and for civilian use are the reduction in size, weight, power, and cost compared with carrying similar previous generation devices.  Prior to the introduction of the smart phone, the end user would need an MP3 player, a hand held GPS, a cellular phone, and a laptop in order to get the same functionality; the size, weight, power, and cost of which were all prohibitive.  Today, all of this functionality slips neatly into your pocket and costs less than \$500.00.

\section*{Evolution of the Technology}
% How has the technology evolved over its history?  Research when the technology was first introduced, how it was introduced, and its major milestones over the technology�s lifetime.

\section*{Technologically Enabling Advances}
% What technical advances in related or component technologies enable this technology?  What knowledge is associated with the development of the technology? (e.g., mechanical knowledge, nuclear, �)
Smart phone technology is really a system of systems, bringing together a cadre of sensors, cellular, and WiFi capability (plus more recently near field communications, or NFC) under an integrated and unified user interface which is facilitated by touch screen capability.  The technology enabling advances have principally been increases battery life, reductions in power draw, and the ability to produce inexpensive, high resolution capacitive touch screens.

\section*{Variants of the Technology}
% Are there variants of the same technology?  If so, what are the differences/similarities of those variants?  (e.g., automobile technology, there are variants in engine type, transmission, etc.).  Variants indicate either different needs, different prioritization of needs, or different solution approaches to the same problem.

\section*{Technology Diffusion}
% Explain the mechanisms and timing of how the technology diffused through the market, military, or society?  How long did it take to be adopted?

\section*{Growth Rate}
% What has been the growth rate of the technology?  Define appropriate measures of effectiveness and/or performance and determine its growth rate.  For example, for automobiles you could look at mpg, average service life, reliability, and other relevant measures.

\section*{Forecast}
% Forecast likely scenarios of where the technology will be 10 � 20 years from today; and/or create a technology roadmap of where the main stakeholders would likely want the technology to be.


\end{document}